\documentclass[a4paper,10pt]{scrartcl}
\usepackage{ucs}
\usepackage[utf8x]{inputenc}
\usepackage{url}
\usepackage[left=3cm,right=3cm,top=2cm,bottom=2cm,nohead]{geometry}
\usepackage[spanish]{babel}
\usepackage[colorlinks=false,pdfborder=0 0 0]{hyperref}

%opening
\title{YajHFC FAQ}
\author{}
\date{}
\begin{document}

\maketitle

\tableofcontents

\section{Instalación}
\subsection{¿Dónde puedo conseguir un visor de faxes para Windows?}

Generalmente puedes utilizar cualquier programa que permita mostrar ficheros TIFF, aunque algunos de ellos muestran los faxes en una resolución baja con una anchura (a la mitad) incorrecta.

Afortunadamente, las versiones recientes de Windows vienen con un programa 
que hace ésto correctamente:

\begin{description}
\item[Windows 95/98/ME/2000:]
Puedes utilizar el programa que se encuentra en Accesorios / Imagen. 
Para utilizarlo con YajHFC, utiliza el botón de exploración en \texttt{Línea de comandos 
para el visor de faxes} y selecciona la ubicación de kodakimg.exe en tu disco duro.

\item[Windows XP:]
Puedes utilizar el programa integrado "Visor de imágenes y faxes". 
Para utilizarlo, introduce el siguiente texto en "Línea de comandos para el visor de faxes":\\
      \verb#rundll32.exe shimgvw.dll,ImageView_Fullscreen %s#
 \end{description}

      
\subsection{¿Dónde puedo obtener un visor de faxes para Linux/*BSD/... ?}
\texttt{kfax} funciona muy bien, pero al igual que en Windows, puedes utilizar cualquier programa que pueda mostrar archivos TIFF, tan sólo busca en la base de paquetes de tu distribución (xloadimage \emph{no} funciona).


\subsection{¿Qué puedo utilizar como visor de faxes en Mac OS X?}
Sólo tienes que introducir \verb#open %s# en \texttt{Línea de comandos para el visor de faxes".}
Los faxes deberían abrirse con la aplicación predeterminada. 
Gracias a Scott Harris por esta ayuda.


\subsection{¿Dónde puedo obtener un visor PostScript?}
\begin{description}
\item[Windows:] Utiliza Ghostview disponible en
\url{http://www.cs.wisc.edu/~ghost/}\\
\textbf{Nota:} también es posible utilizar el programa Acrobat Distiller
(verificado sólo con la versión completa) que permite abrir los ficheros PS
y visualizarlo al instante. Para ello hay que introducir la ruta al programa
en \texttt{Línea de comandos para el visor archivos PostScript},
generalmente:
\verb#"C:\Archivos de programa\Adobe\Acrobat 5.0\Distillr\Acrodist.exe" %s#

\item[Linux/*BSD/...:] Simplemente instala uno de los paquetes de visores
PostScript
(por ejemplo: \texttt{gv, kghostview, gnome-gv, ...})
\end{description}


\section{Utilización del programa}	

\subsection{¿Cómo puedo editar las plantillas de las portadas?}
Las plantillas tienen que estar en el mismo formato PostScript especial 
que utiliza el programa \texttt{faxcover} de HylaFAX. Revisa la siguiente ayuda 
para saber cómo crear / modificar este tipo de archivos: \\
\url{http://www.hylafax.org/HylaFAQ/Q202.html}\\
\url{http://www.hylafax.org/howto/tweaking.html}\\

\subsection{Quiero acceder a la agenda telefónica por medio de JDBC pero YajHFC no encuentra el controlador aunque especifique la ruta correcta para llamar a Java.}

Si utilizas el argumento \texttt{-jar}, Java ignora la ruta definida por el usuario.
Por tanto inicia YajHFC utilizando el siguiente comando (reemplaza \texttt{/ruta/a/controladorbdd.jar} y \texttt{/ruta/a/yajhfc.jar} con sus respectivas rutas reales y nombre de fichero (por supuesto):
\begin{description}
\item [Linux/Unix:] \verb#java -classpath /ruta/a/controaldorbdd.jar:/ruta/a/yajhfc.jar yajhfc.Launcher#
\item [Windows:] \verb#java -classpath c:\ruta\a\controladorbdd.jar;c:\ruta\a\yajhfc.jar yajhfc.Launcher#
\end{description}

\subsection{¿Por qué no se puede seleccionar (gris claro) la opción de \texttt{Ver->Sólo faxes propios}?}

Actualmente YajHFC sólo puede filtrar las columnas visibles. Por lo que tienes que añadir la columna de "Propietario" a la tabla en la que quieres filtrar por propietario.

\subsection{¿Por qué no se puede seleccionar (gris claro) la opción de \texttt{Ver->Mostrar trabajos con errores}?}

Por la misma razón que el apartado anterior: actualmente, YajHFC sólo puede trabajar con las columnas que están visibles.
Para mostrar los trabajos de fax erróneos es necesario que las siguientes columnas sean visibles:
\begin{description}
\item[Pestaña Recibidos:] \texttt{Descripción del error}
\item[Pestaña Enviados / Transmisión:] \texttt{Estado del trabajo} \emph{or} \texttt{Estado}
\end{description}

\subsection{¿Qué puedo poner como valor \texttt{igual a} en el cuadro de diálogo de filtrado personalizado? }

Expresiones regulares. Puedes encontrar una breve referencia sobre la sintaxis permitida en
\url{http://java.sun.com/j2se/1.5.0/docs/api/java/util/regex/Pattern.html}

ten en cuenta que las expresiones regulares no son lo mismo que los caracteres comodín:
Por ejemplo, para obtener el mismo efecto que el comodín \verb.*. tienes que utilizar \verb#.*#
y para simular el efecto de \verb#?# hay que utilizar \verb#.#.

\subsection{¿Qué argumentos de línea de comandos entiende YajHFC?}

\begin{verbatim}
Uso general:
java -jar yajhfc.jar [--help] [--debug] [--admin] [--background|--noclose]
         [--configdir=directory] [--loadplugin=filename]
         [--showtab=0|R|1|S|2|T] [--recipient=...] [--stdin | filename ...]
Descripción de los argumentos:
filename     El nombre del archivo PostScript para enviar.
--stdin      Lee el archivo para enviar de una entrada estándar
--recipient  Specifies the phone number of a recipient to send the fax to.
             You may specify multiple arguments for multiple recipients.
--admin      Inicia en modo administrador
--debug      Salida de información de errores
--background Si no hay ninguna instancia en ejecución, iniciar una nueva instancia
             y terminar (después de enviar el archivo)
--noclose    No cerrar YajHFC después de enviar el fax
--showtab    Sets the tab to display on startup. Specify 0 or R for the "Received",
             1 or S for the "Sent" or 2 or T for the "Transmitting" tab.
--loadplugin Specifies the jar file of a YajHFC plugin to load
--configdir  Establece un directorio de configuración para utilizar en lugar de ~/.yajhfc
--help       Muestra este texto
\end{verbatim}


\subsection{Cuando intento ver los faxes enviados siempre obtengo un mensaje de error 
   que dice \texttt{Formato de archivo PCL no soportado}, aunque todos los documentos 
   están el formato PostScript/PDF.}

Verifica que la opción \texttt{Utilizar fichero PCL para la corrección de error} en el 
cuadro de Opciones está seleccionada e inténtalo de nuevo.
 
Algunas versiones de HylaFAX reportan 
un tipo de fichero "PCL" para todos los documentos asociados con un trabajo. 
Si esta opción está seleccionada, YajHFC intenta adivinar el tipo de fichero 
utilizando la extensión si se reporta PCL (lo cual generalmente funciona bastante bien).

\subsection{Cuando YajHFC se ejecuta en Windows algunas veces guarda su configuración en \texttt{C:\textbackslash .yajhfc} en lugar de \texttt{C:\textbackslash Documents and Settings\textbackslash USERNAME\textbackslash .yajhfc}}

De manera predeterminada, YajHFC guarda su configuración en el subdirectorio \texttt{.yajhfc} del directorio que devuelve
la propiedad del sistema Java \texttt{user.home}.
A veces, algunas versiones de Java parece que no establecen esta propiedad correctamente, lo que provoca dicho error.

Para solucionar ésto, puedes establecer esta propiedad explícitamente cuando inicias YajHFC utilizando la línea de comandos de java \texttt{-D}, por ejemplo:\\
\texttt{java -Duser.home=\%USERPROFILE\% -jar "C:\textbackslash Archivos de Programa\textbackslash yajhfc.jar"}

\subsection{¿Qué significa la columna XYZ?}

Yo tampoco lo sé exactamente porque la descripción de las columnas ha sido copiada del manual de  \verb#faxstat(1)# (JobFmt/RcvFmt), abreviada y traducida

\section{Varios}

\subsection{¿Por qué se guardan las contraseñas en texto plano?}

Simplemente porque no hay otro método que sea mejor.


YajHFC puede codificar/"encriptar" las contraseñas antes de almacenarlas, 
pero si lo hace siempre es posible visualizar el código fuente para encontrarlas 
y desencriptarlas (incluso aunque YajHFC fuera de código cerrado podrías reventarlas o experimentar un poco sobre cómo hacerlo).


El único mitodo seguro requeriría introducir una contraseña maestra cada vez que se inicia YajHFC, pero en mi opinión no es mejor que introducir la contraseña real.


\subsection{¿Por qué escogiste ese nombre tan tonto?}

YajHFC empezó como "proyecto de prueba" para Java y la biblioteca \texttt{gnu.hylafax}
y por lo tanto no tuvo un nombre "bonito". Después de trabajar un poco sobre ello, me di cuenta de que se volvería útil, por lo que elegí darle un nombre.
Dado que estaba con la herramienta Yast de SuSE en ese momento y sabía 
que había / hay otros clientes Java para HylaFAX, lo llamé "\textbf{y}et \textbf{a}nother \textbf{J}ava \textbf{H}ylaFAX \textbf{c}lient" (en español, "otro cliente Java para HylaFAX").
 

\end{document}
