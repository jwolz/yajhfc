\documentclass[a4paper,10pt]{scrartcl}
\usepackage{ucs}
\usepackage[utf8x]{inputenc}
\usepackage{url}
\usepackage[left=3cm,right=3cm,top=2cm,bottom=2cm,nohead]{geometry}
\usepackage[spanish]{babel}
\usepackage[colorlinks=false,pdfborder=0 0 0]{hyperref}
\usepackage[T1]{fontenc}
\usepackage{textcomp}

%opening
\title{YajHFC FAQ}
\author{}
\date{}
\begin{document}
\sloppy

\maketitle

\tableofcontents

\section{Instalación}
\subsection{¿Dónde puedo conseguir un visor de faxes para Windows?}

Generalmente puedes utilizar cualquier programa que permita mostrar archivos TIFF, aunque algunos de ellos muestran los faxes en una resolución baja con una anchura (a la mitad) incorrecta.

Afortunadamente, las versiones recientes de Windows vienen con un programa 
que hace ésto correctamente:

\begin{description}
\item[Windows 95/98/ME/2000:]
Puedes utilizar el programa que se encuentra en Accesorios / Imagen. 
Para utilizarlo con YajHFC, utiliza el botón de exploración en \texttt{Línea de comandos 
para el visor de faxes} y selecciona la ubicación de kodakimg.exe en tu disco duro.

\item[Windows XP:]
Puedes utilizar el programa integrado "Visor de imágenes y faxes". 
Para utilizarlo, introduce el siguiente texto en "Línea de comandos para el visor de faxes":\\
      \verb#rundll32.exe shimgvw.dll,ImageView_Fullscreen %s#
 \end{description}

      
\subsection{¿Dónde puedo obtener un visor de faxes para Linux/*BSD/... ?}
\texttt{kfax} funciona muy bien, pero al igual que en Windows, puedes utilizar cualquier programa que pueda mostrar archivos TIFF, tan sólo busca en la base de paquetes de tu distribución (xloadimage \emph{no} funciona).


\subsection{¿Qué puedo utilizar como visor de faxes en Mac OS X?}
Sólo tienes que introducir \verb#open %s# en \texttt{Línea de comandos para el visor de faxes".}
Los faxes deberían abrirse con la aplicación predeterminada. 
Gracias a Scott Harris por esta ayuda.


\subsection{¿Dónde puedo obtener un visor PostScript?}
\begin{description}
\item[Windows:] Utiliza Ghostview disponible en
\url{http://www.cs.wisc.edu/~ghost/}\\
\textbf{Nota:} también es posible utilizar el programa Acrobat Distiller
(verificado sólo con la versión completa) que permite abrir los archivos PS
y visualizarlo al instante. Para ello hay que introducir la ruta al programa
en \texttt{Línea de comandos para el visor archivos PostScript},
generalmente:
\verb#"C:\Archivos de programa\Adobe\Acrobat 5.0\Distillr\Acrodist.exe" %s#

\item[Linux/*BSD/...:] Simplemente instala uno de los paquetes de visores
PostScript
(por ejemplo: \texttt{gv, kghostview, gnome-gv, ...})
\end{description}


\subsection{Where can I get GhostScript from?}
\begin{description}
\item[Windows:] Download it from \url{http://www.cs.wisc.edu/~ghost/}
\item[Linux/*BSD/...:] Install the GhostScript package for your distribution (on most installations this package will already be installed; if not: the package name usually begins with \texttt{ghostscript} or \texttt{gs})
\end{description}

\subsection{Where can I get TIFF2PDF from?}
\begin{description}
\item[Windows:] Download it from \url{http://gnuwin32.sourceforge.net/downlinks/tiff.php}\\ If this link does not work, please see \url{http://gnuwin32.sourceforge.net/packages/tiff.htm} or \url{http://www.libtiff.org/}.
\item[Linux/*BSD/...:] Install the libtiff tools package for your distribution. Usually this package will have the word \texttt{tiff} in its name (on Debian/Ubuntu it is called \texttt{libtiff-tools} and on SUSE \texttt{tiff}).
\end{description}

\section{Utilización del programa}	

\subsection{¿Cómo puedo editar las plantillas PostScript de las portadas?}
Las plantillas tienen que estar en el mismo formato PostScript especial 
que utiliza el programa \texttt{faxcover} de HylaFAX. Revisa la siguiente ayuda 
para saber cómo crear / modificar este tipo de archivos: \\
\url{http://www.hylafax.org/HylaFAQ/Q202.html}\\
\url{http://www.hylafax.org/howto/tweaking.html}\\

Como alternativa, a partir de YajHFC 0.3.7, puedes utilizar páginas de portada en HTML o bien en formato XSL:FO u ODT (OpenDocument Text), mediante un complemento.



\subsection{¿Qué campos se reconocen en una página de portada HTML?}

Las siguientes ``palabras'' se reemplazan (sin tener en cuenta mayúsculas / minúsculas) con sus valores correspondientes cuando se utiliza un archivo HTML como página de portada:

\begin{center}
\begin{tabular}{|l|l|}
\hline
\bfseries Palabra & \bfseries Significado \\
\hline\hline
\ttfamily @@Name@@ & Nombre del destinatario \\\hline
\ttfamily @@Location@@ & Ubicación del destinatario \\\hline
\ttfamily @@Company@@ & Empresa del destinatario \\\hline
\ttfamily @@Faxnumber@@ & Número de fax del destinatario \\\hline
\ttfamily @@Voicenumber@@ & Número de teléfono del destinatario \\\hline
\ttfamily @@FromName@@ & Nombre del emisor \\\hline
\ttfamily @@FromLocation@@ & Ubicación del emisor \\\hline
\ttfamily @@FromCompany@@ & Empresa del emisor \\\hline
\ttfamily @@FromFaxnumber@@ & Número de fax del emisor \\\hline
\ttfamily @@FromVoicenumber@@ & Número de teléfono del emisor \\\hline
\ttfamily @@FromEMail@@ & Dirección de correo electrónico del emisor \\\hline
\ttfamily @@Subject@@ & Asunto del fax \\\hline
\ttfamily @@Date@@ & Fecha actual \\\hline
\ttfamily @@PageCount@@ & Número de páginas sin incluir la página de portada \\\hline
\ttfamily @@Comments@@ & Comentarios para el fax \\\hline
\end{tabular}
\end{center}

Starting with version 0.4.0 the following additional fields are available (note that the \textit{italic} fields will be empty when you use resend fax):
\begin{center}
\begin{tabular}{|l|p{.7\textwidth}|}
\hline
\bfseries Word & \bfseries Meaning \\
\hline\hline
\ttfamily @@Surname@@ & The recipient's surname (resent faxes: the same as \texttt{@@Name@@}) \\\hline
\ttfamily\itshape @@GivenName@@ & The recipient's given name \\\hline
\ttfamily\itshape @@Title@@ & The recipient's title \\\hline
\ttfamily\itshape @@Position@@ & The recipient's position \\\hline
\ttfamily\itshape @@Department@@ & The recipient's department\\\hline
\ttfamily @@CompanyName@@ & The recipient's company name (without department) (resent faxes: the same as \texttt{@@Company@@})\\\hline
\ttfamily\itshape @@Street@@ & The recipient's street name \\\hline
\ttfamily @@Place@@ & The recipient's location (without street or ZIP code) (resent faxes: the same as \texttt{@@Location@@})\\\hline
\ttfamily\itshape @@ZIPCode@@ & The recipient's ZIP Code \\\hline
\ttfamily\itshape @@State@@ & The recipient's state\\\hline
\ttfamily\itshape @@Country@@ & The recipient's country\\\hline
\ttfamily\itshape @@EMail@@ & The recipient's e-mail address\\\hline
\ttfamily\itshape @@WebSite@@ & The recipient's website\\\hline\hline
\ttfamily @@FromSurname@@ & The sender's  surname \\\hline
\ttfamily @@FromGivenName@@ & The sender's  given name \\\hline
\ttfamily @@FromTitle@@ & The sender's  title \\\hline
\ttfamily @@FromPosition@@ & The sender's position \\\hline
\ttfamily @@FromDepartment@@ & The sender's  department\\\hline
\ttfamily @@FromCompanyName@@ & The sender's  company name (without department)\\\hline
\ttfamily @@FromStreet@@ & The sender's  street name \\\hline
\ttfamily @@FromPlace@@ & The sender's  location (without street or ZIP code)\\\hline
\ttfamily @@FromZIPCode@@ & The sender's  ZIP Code \\\hline
\ttfamily @@FromState@@ & The sender's  state\\\hline
\ttfamily @@FromCountry@@ & The sender's  country\\\hline
\ttfamily @@FromEMail@@ & The sender's  e-mail address\\\hline
\ttfamily @@FromWebSite@@ & The sender's website\\\hline
\ttfamily @@TotalPageCount@@ & The number of pages \textit{including} the cover page \\\hline
\end{tabular}
\end{center}

El reemplazo se realiza a nivel de código fuente, por lo que esas palabras no se reconocerán en los cambios de formato dentro del mismo (por ejemplo, \texttt{@@sub\textit{ject@@}}).

\subsection{Me gustaba más el antiguo cuadro de diálogo. ¿Puedo volver a utilizarlo?}

Simplemente abre el cuadro de diálogo de Opciones y selecciona \texttt{Tradicional} en ``Estilo para el cuadro de diálogo de envío''.

\subsection{Quiero acceder a la agenda telefónica por medio de JDBC pero YajHFC no encuentra el controlador aunque especifique la ruta correcta para llamar a Java.}

Si utilizas el argumento \texttt{-jar}, Java ignora la ruta definida por el usuario.
Por tanto inicia YajHFC utilizando el siguiente comando (reemplaza \texttt{/ruta/a/controladorbdd.jar} y \texttt{/ruta/a/yajhfc.jar} con sus respectivas rutas reales y nombre de archivo (por supuesto):
\begin{description}
\item [Linux/Unix:] \verb#java -classpath /ruta/a/controaldorbdd.jar:/ruta/a/yajhfc.jar yajhfc.Launcher#
\item [Windows:] \verb#java -classpath c:\ruta\a\controladorbdd.jar;c:\ruta\a\yajhfc.jar yajhfc.Launcher#
\end{description}

\subsection{¿Qué puedo poner como valor \texttt{igual a} en el cuadro de diálogo de filtrado personalizado? }

Expresiones regulares. Puedes encontrar una breve referencia sobre la sintaxis permitida en
\url{http://java.sun.com/j2se/1.5.0/docs/api/java/util/regex/Pattern.html}

Ten en cuenta que las expresiones regulares no son lo mismo que los caracteres comodín:
Por ejemplo, para obtener el mismo efecto que el comodín \verb.*. tienes que utilizar \verb#.*#
y para simular el efecto de \verb#?# hay que utilizar \verb#.#.

\subsection{¿Qué argumentos de línea de comandos entiende YajHFC?}

\begin{verbatim}
Uso general:
java -jar yajhfc.jar [--help] [--debug] [--admin] [--background|--noclose]
         [--configdir=directory] [--loadplugin=filename] [--logfile=filename]
         [--showtab=0|R|1|S|2|T] [--recipient=...] [--stdin | filename ...]
Descripción de los argumentos:
filename     El nombre del archivo PostScript para enviar.
--stdin      Lee el archivo para enviar de una entrada estándar
--recipient  Especifica el número de teléfono del destinatario para enviar un fax.
             Es posible especificar varios valores para múltiples destinatarios.
--admin      Inicia en modo administrador
--debug      Salida de información de errores
--logfile    Archivo de registro donde registrar la información de depuración (si no se especifica, usar la salida estándar "stdout")
--background Si no hay ninguna instancia en ejecución, iniciar una nueva instancia
             y terminar (después de enviar el archivo)
--noclose    No cerrar YajHFC después de enviar el fax
--showtab    Establece la pestaña para mostrar al inicio. Especifica 
             0 o R para la pestaña "Recibidos", 1 o S para "Enviados" o
             2 o T para "Transmistiendo".
--loadplugin Especifica el archivo jar de un plugin de YajHFC para cargar
--no-plugins Descativa la carga de complementos desde el archivo plugin.lst.
--configdir  Establece un directorio de configuración para utilizar en lugar de ~/.yajhfc
--help       Muestra este texto
\end{verbatim}

\subsection{How can I give cover page information using the \texttt{-{-}recipient} switch?}

Starting with version 0.4.0, you can give that information using \texttt{name:value} pairs, separated by semicolons. For example, to send a fax to ``John Doe'' in ``Example town'' with the fax number 0123456, use the following command line:

\texttt{java -jar yajhfc.jar \textit{[...]} -{-}recipient="givenname:John;surname:Doe;location:Example Town;faxnumber:0123456" \textit{[...]}}

The following field names are recognized:
\begin{center}
\begin{tabular}{|l|p{.7\textwidth}|}
\hline
\bfseries Field name & \bfseries Meaning \\
\hline\hline
\ttfamily surname & The recipient's surname\\\hline
\ttfamily givenname & The recipient's given name \\\hline
\ttfamily title & The recipient's title \\\hline
\ttfamily position & The recipient's position \\\hline
\ttfamily department & The recipient's department\\\hline
\ttfamily company & The recipient's company name\\\hline
\ttfamily street & The recipient's street name \\\hline
\ttfamily location & The recipient's location\\\hline
\ttfamily zipcode & The recipient's ZIP Code \\\hline
\ttfamily state & The recipient's state\\\hline
\ttfamily country & The recipient's country\\\hline
\ttfamily email & The recipient's e-mail address\\\hline
\ttfamily faxnumber & The recipient's fax number \\\hline
\ttfamily voicenumber & The recipient's voice number \\\hline
\ttfamily website & The recipient's website\\\hline
\end{tabular}
\end{center}

\subsection{¿Qué significa la columna XYZ?}

Yo tampoco lo sé exactamente porque la descripción de las columnas ha sido copiada del manual de  \verb#faxstat(1)# (JobFmt/RcvFmt), abreviada y traducida

\subsection{How can I specify some default settings?}

Starting with version 0.4.0 the following files (if they exist) are loaded in order to retrieve the saved settings:
\begin{enumerate}
 \item \texttt{[Directory where yajhfc.jar resides]/settings.default}
 \item the user settings from \texttt{\{user.home\}\footnote{In Windows \texttt{user.home} usually is \texttt{C:\textbackslash Documents and Settings\textbackslash USERNAME}.}/.yajhfc/settings} (if you specified \texttt{-{-}configdir=DIR}, \texttt{DIR/settings} is used instead)
 \item \texttt{[Directory where yajhfc.jar resides]/settings.override}
\end{enumerate}

Settings from files loaded later override the settings specified in the files loaded earlier, i.e. the settings from \texttt{settings.override} take precedence over those in the user settings and \texttt{settings.default}.
\medskip

This logic can be used to specify default settings (e.g. in a networked environment): \\
Simply configure one YajHFC installation as you like, then copy \texttt{\{user.home\}/.yajhfc/settings} to the directory where \texttt{yajhfc.jar} is and rename it to \texttt{settings.default}.
\medskip

Overrides can be specified in a similar fashion. In this case, you are encouraged to edit the settings file (it is a plain text file), and remove any lines that specify settings you wish not to override (usually you will want to remove at least \texttt{user} and \texttt{pass-obfuscated} (user name and password used to connect to the HylaFAX server), \texttt{FromName}, \texttt{*ColState} (the widths of the table columns), \texttt{*Bounds} (the position of the different windows) and \texttt{mainwinLastTab}).

Note, however that the user can still set these settings to a different value for a running YajHFC instance. They are reset to the override values only when YajHFC is started again (in other words: the user can set these settings, but they are not saved between different runs of YajHFC).


\subsection{How can I display the HylaFAX \texttt{archive} folder in YajHFC?}

With version 0.4.0 support for the HylaFAX \texttt{archive} folder has been implemented.

This directory cannot be accessed using the "`normal"' HylaFAX connection like the other folders, however, because HylaFAX does only allow to get the list of subfolders in the \texttt{archive} directory, but not to get any information (except the ID) about the archived jobs or the attached documents (if you know a HylaFAX version where this is different, please let me know).

For this reason the directory has to be accessed by other methods. Currently (0.4.0) YajHFC only supports access using the file system. This means you have to export the  \texttt{archive} directory on the server using Samba, NFS or any other network file system, mount it on the client (if you use Unix; on Windows, you can just use UNC paths) and tell YajHFC in the options under which path the \texttt{archive} directory can be found.

When you have done that, the archive table should work just like the other tables do.

\subsection{What do the different options under \texttt{Paths \& Viewers -> View and send settings} do (0.4 and up)?}

For the impatient: The recommended settings are (they are not the default, because they require \texttt{gs} and \texttt{tiff2pdf}):
\begin{itemize}
 \item \textbf{Format:} PDF or TIFF
 \item \textbf{Send multiple files as:} Single file except cover
 \item \textbf{View as single file:} Yes
 \item \textbf{Always view/send in this format:} Yes
\end{itemize}

\subsubsection{Format for viewing/sending}

The format to which the documents should be converted if necessary. Generally PDF and TIFF give better results than PostScript here (as the latter uses GhostScript's \texttt{pswrite} device).

\subsubsection{Send multiple files as}

{\parindent0pt
\textbf{Multiple files:}\\
Same behaviour as pre-0.4.0 versions. If you attach multiple documents to a single fax job, these documents are converted to PS or PDF, but kept as separate files (e.g. if you send a fax with files \texttt{doc.ps} and \texttt{picture.jpg}, two separate files are uploaded)
\medskip

\textbf{Single file except for cover:}\\
One file is created for the whole fax, but the cover page is kept as a separate file (e.g. if you send a fax with \texttt{doc.ps} and \texttt{picture.jpg}, a single PDF/PS/TIFF file is created and uploaded).\\
\textit{Advantage:} The resolution is reduced to 196dpi (\textrightarrow smaller files/upload) and a single document file can be used when sending a fax to multiple destinations.
\medskip

\textbf{Complete fax as single file:}\\
One file is created for the whole fax including the cover page. If the fax has no cover this behaves identical to the above case.\\
\textit{Advantage:} No conversions on the client are necessary when viewing the sent fax.\\
\textit{Disadvantage:} When sending a fax to multiple recipients one file per recipient has to be created and uploaded.
}

\subsubsection{View faxes as single file}
If this option is checked and a fax on the server consists of multiple files, a single file is created (on the client) for viewing.


\subsubsection{View/send faxes always in this format}
This option modifies the behaviour of ``Send multiple files as'' and ``View faxes as single file''.\\
When this option is \textit{unchecked}, a fax is only converted if it consists of multiple files. If it consists of only a single file, the format is left as it is.\\
When this option is \textit{checked}, a fax consisting of a single file is also converted when that single file has a different format than the one selected at ``Format for sending/viewing''.\\
\textit{Advantage:} A single viewer is used for both sent and received faxes (e.g. to view received faxes as PDF).\\
\textit{Disadvantage:} Usually more format conversions are necessary on the client.

\subsection{Which characters are recognized as date/time format?}

The date is formatted using a Java \texttt{SimpleDateFormat}. A description of the characters recognized can be found at \url{http://java.sun.com/j2se/1.5.0/docs/api/java/text/SimpleDateFormat.html}.

\section{Problems/Known bugs}

\subsection{¡He creado un documento / página de portada en HTML pero el formato en YajHFC se ve mal!}

YajHFC utiliza el soporta HTML que está integrado en Java (\texttt{HTMLEditorKit} / \texttt{HTMLDocument}) para convertir el HTML en PostScript. Es un soporte muy limitado ya que sólo soporta la especificación HTML 3.2.\\
Esto significa que los diseños complejos seguramente no se rendericen correctamente en YajHFC.
Para lograr el diseño que quieres, tienes básicamente las siguientes alternativas:

\begin{itemize}
 \item Utilizar el método de ``prueba \& error'' hasta que el diseño quede bien (el botón de previsualización en el cuadro de diálogo de envío mostrará el diseño del HTML convertido).
 \item Utilizar un editor HTML como Ekit (\url{http://www.hexidec.com/ekit.php}) que también utiliza el soporte HTML de Java, por lo que el documento se renderizará de manera similar en YajHFC.
 \item Utilizar otro formato para la página de portada (como XSL:FO u ODT con el complemento FOP).
\end{itemize}

\subsection{Cuando intento ver los faxes enviados siempre obtengo un mensaje de error 
   que dice \texttt{Formato de archivo PCL no soportado}, aunque todos los documentos 
   están el formato PostScript/PDF.}

Verifica que la opción \texttt{Utilizar archivo PCL para la corrección de error} en el 
cuadro de Opciones está seleccionada e inténtalo de nuevo.
 
Algunas versiones de HylaFAX reportan 
un tipo de archivo "PCL" para todos los documentos asociados con un trabajo. 
Si esta opción está seleccionada, YajHFC intenta adivinar el tipo de archivo 
utilizando la extensión si se reporta PCL (lo cual generalmente funciona bastante bien).

\subsection{A menudo obtengo un error cuando envío dos o más faxes en una fila. ¿Qué puedo hacer?}

Parece que algunas versiones del servidor HylaFAX tienen problemas cuando se envía más de un fax en cada sesión.

Para solucionar este problema, hay que ir a la pestaña \texttt{Servidor} en el cuadro de diálogo Opciones, activar la casilla \texttt{Crear una nueva sesión por cada acción} y comprobar si el problema persiste.
Si hacer ésto \emph{no} soluciona el problema, por favor, envíame un correo electrónico con el informe del error.

\subsection{Cuando YajHFC se ejecuta en Windows algunas veces guarda su configuración en \texttt{C:\textbackslash .yajhfc} en lugar de \texttt{C:\textbackslash Documents and Settings\textbackslash USERNAME\textbackslash .yajhfc}}

De manera predeterminada, YajHFC guarda su configuración en el subdirectorio \texttt{.yajhfc} del directorio que devuelve
la propiedad del sistema Java \texttt{user.home}.
A veces, algunas versiones de Java parece que no establecen esta propiedad correctamente, lo que provoca dicho error.

Para solucionar ésto, puedes establecer esta propiedad explícitamente cuando inicias YajHFC utilizando la línea de comandos de java \texttt{-D}, por ejemplo:\\
\texttt{java -Duser.home=\%USERPROFILE\% -jar "C:\textbackslash Archivos de Programa\textbackslash yajhfc.jar"}

\subsection{The tray icon is not shown!}

Starting with version 0.4.0 YajHFC supports a tray icon which will be shown when you run YajHFC under Java 1.6 (``Java 6'').
If you use Java 1.5 (``Java 5''), a tray icon is not supported.

So, please make sure you have Java 1.6 installed. If you are absolutely sure you have Java 1.6 installed and the tray icon is still not shown, please mail me a bug report.


\section{Varios}

\subsection{¿Por qué se guardan las contraseñas en texto plano?}

Simplemente porque no hay otro método que sea mejor.


YajHFC puede codificar/"cifrar" las contraseñas antes de almacenarlas, 
pero si lo hace siempre es posible visualizar el código fuente para encontrarlas 
y descifrarlas (incluso aunque YajHFC fuera de código cerrado podrías reventarlas o experimentar un poco sobre cómo hacerlo).


El único método seguro requeriría introducir una contraseña maestra cada vez que se inicia YajHFC, pero en mi opinión no es mejor que introducir la contraseña real.

Because of many requests passwords are obfuscated using a simple algorithm in version 0.4.0 and up.
The statement above still holds true, however, i.e. once you read the source code, the password can be decrypted easily.

\subsection{¿Por qué escogiste ese nombre tan tonto?}

YajHFC empezó como "proyecto de prueba" para Java y la biblioteca \texttt{gnu.hylafax}
y por lo tanto no tuvo un nombre "bonito". Después de trabajar un poco sobre ello, me di cuenta de que se volvería útil, por lo que elegí darle un nombre.
Dado que estaba con la herramienta Yast de SuSE en ese momento y sabía 
que había / hay otros clientes Java para HylaFAX, lo llamé "\textbf{y}et \textbf{a}nother \textbf{J}ava \textbf{H}ylaFAX \textbf{c}lient" (en español, "otro cliente Java para HylaFAX").
 

\end{document}
