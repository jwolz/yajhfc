\documentclass[a4paper,10pt]{scrartcl}
\usepackage{ucs}
\usepackage[utf8x]{inputenc}
\usepackage{url}
\usepackage[left=3cm,right=3cm,top=2cm,bottom=2cm,nohead]{geometry}
\usepackage[spanish]{babel}
\usepackage[colorlinks=false,pdfborder=0 0 0]{hyperref}
\usepackage[T1]{fontenc}
\usepackage{textcomp}

%opening
\title{YajHFC FAQ}
\author{}
\date{}
\begin{document}
\sloppy

\maketitle

\tableofcontents

\section{Instalación}
\subsection{¿Dónde puedo conseguir un visor de faxes para Windows?}

Generalmente puede utilizar cualquier programa que permita mostrar archivos TIFF, aunque algunos de ellos muestran los faxes en una resolución baja con una anchura incorrecta (a la mitad).

Afortunadamente, las versiones recientes de Windows vienen con un programa 
que hace ésto correctamente:

\begin{description}
\item[Windows 95/98/ME/2000:]
Puede utilizar el programa que se encuentra en Accesorios / Imagen. 
Para utilizarlo con YajHFC, use el botón de exploración en \texttt{Línea de comandos 
para el visor de faxes} y seleccione la ubicación de kodakimg.exe en su disco duro.

\item[Windows XP:]
Puede utilizar el programa integrado "Visor de imágenes y faxes". 
Para utilizarlo, introduzca el siguiente texto en "Línea de comandos para el visor de faxes":\\
      \verb#rundll32.exe shimgvw.dll,ImageView_Fullscreen %s#
 \end{description}

      
\subsection{¿Dónde puedo obtener un visor de faxes para Linux/*BSD/... ?}
\texttt{kfax} funciona muy bien, pero al igual que en Windows, puede utilizar cualquier programa que pueda mostrar archivos TIFF, tan sólo busque en la base de paquetes de su distribución (xloadimage \emph{no} funciona).


\subsection{¿Qué puedo utilizar como visor de faxes en Mac OS X?}
Sólo tiene que introducir \verb#open %s# en \texttt{Línea de comandos para el visor de faxes".}
Los faxes deberían abrirse con la aplicación predeterminada. 
Gracias a Scott Harris por esta ayuda.


\subsection{¿Dónde puedo obtener un visor PostScript?}
\begin{description}
\item[Windows:] Utilice Ghostview disponible en
\url{http://www.cs.wisc.edu/~ghost/}\\
\item[Linux/*BSD/...:] Simplemente instale uno de los paquetes de visores PostScript
		(por ejemplo: \texttt{gv, kghostview, gnome-gv, ...})
\end{description}


\subsection{¿Dónde puedo obtener GhostScript?}
\begin{description}
\item[Windows:] Puede descargarlo desde \url{http://www.cs.wisc.edu/~ghost/}
\item[Linux/*BSD/...:] Puede instalar el paquete GhostScript para su distribución (en la mayoría de las instalaciones este paquete ya estará instalado; si no lo está: el nombre del paquete comienza, normalmente, por \texttt{ghostscript} o \texttt{gs})
\end{description}

\subsection{¿Dónde puedo obtener TIFF2PDF?}
\begin{description}
\item[Windows:] Puede descargarlo desde \url{http://gnuwin32.sourceforge.net/downlinks/tiff.php}\\ Si el enlace no funciona, consulte \url{http://gnuwin32.sourceforge.net/packages/tiff.htm} o \url{http://www.libtiff.org/}.
\item[Linux/*BSD/...:] Puede instalar el paquete de herramientas libtiff para su distribución. Normalmente este paquete contiene el término \texttt{tiff} en su nombre (en Debian/Ubuntu se llama \texttt{libtiff-tools} y en openSUSE \texttt{tiff}).
\end{description}

\section{Utilización del programa}	

\subsection{¿Cómo puedo editar las plantillas PostScript de las portadas?}
Las plantillas tienen que estar en el mismo formato PostScript especial 
que utiliza el programa \texttt{faxcover} de HylaFAX. Revise la siguiente ayuda 
para saber cómo crear / modificar este tipo de archivos: \\
\url{http://www.hylafax.org/HylaFAQ/Q202.html}\\
\url{http://www.hylafax.org/howto/tweaking.html}\\

Como alternativa, a partir de YajHFC 0.3.7, puede utilizar páginas de portada en HTML o bien en formato XSL:FO u ODT (OpenDocument Text), mediante un complemento.

\subsection{¿Qué campos se reconocen en una página de portada HTML?}

Las siguientes ``palabras'' se reemplazan (sin tener en cuenta mayúsculas / minúsculas) con sus valores correspondientes cuando se utiliza un archivo HTML como página de portada:

\begin{center}
\begin{tabular}{|l|l|}
\hline
\bfseries Palabra & \bfseries Significado \\
\hline\hline
\ttfamily @@Name@@ & Nombre del destinatario \\\hline
\ttfamily @@Location@@ & Ubicación del destinatario \\\hline
\ttfamily @@Company@@ & Empresa del destinatario \\\hline
\ttfamily @@Faxnumber@@ & Número de fax del destinatario \\\hline
\ttfamily @@Voicenumber@@ & Número de teléfono del destinatario \\\hline
\ttfamily @@FromName@@ & Nombre del emisor \\\hline
\ttfamily @@FromLocation@@ & Ubicación del emisor \\\hline
\ttfamily @@FromCompany@@ & Empresa del emisor \\\hline
\ttfamily @@FromFaxnumber@@ & Número de fax del emisor \\\hline
\ttfamily @@FromVoicenumber@@ & Número de teléfono del emisor \\\hline
\ttfamily @@FromEMail@@ & Dirección de correo electrónico del emisor \\\hline
\ttfamily @@Subject@@ & Asunto del fax \\\hline
\ttfamily @@Date@@ & Fecha actual \\\hline
\ttfamily @@PageCount@@ & Número de páginas sin incluir la página de portada \\\hline
\ttfamily @@Comments@@ & Comentarios para el fax \\\hline
\end{tabular}
\end{center}

Desde la versión 0.4.0 se encuentran disponibles los siguientes campos adicionales (tenga en cuenta que los campos en \textit{cursiva} estarán vacíos cuando reenvíe un fax):
\begin{center}
\begin{tabular}{|l|p{.7\textwidth}|}
\hline
\bfseries Palabra & \bfseries Significado \\
\hline\hline
\ttfamily @@Surname@@ & Apellido del destinatario (faxes reenviados: igual que \texttt{@@Name@@}) \\\hline
\ttfamily\itshape @@GivenName@@ & Nombre de pila del destinatario \\\hline
\ttfamily\itshape @@Title@@ & Tratamiento del destinatario \\\hline
\ttfamily\itshape @@Position@@ & Cargo del destinatario \\\hline
\ttfamily\itshape @@Department@@ & Departamento del destinatario\\\hline
\ttfamily @@CompanyName@@ & Nombre de la empresa del destinatario (sin el departamento) (faxes reenviados: igual que \texttt{@@Company@@})\\\hline
\ttfamily\itshape @@Street@@ & Nombre de la calle del destinatario \\\hline
\ttfamily @@Place@@ & Ubicación del destinatario (sin la calle ni el código postal) (faxes reenviados: igual que \texttt{@@Location@@})\\\hline
\ttfamily\itshape @@ZIPCode@@ & Código postal del destinatario \\\hline
\ttfamily\itshape @@State@@ & Estado o provincia del destinatario\\\hline
\ttfamily\itshape @@Country@@ & País del destinatario\\\hline
\ttfamily\itshape @@EMail@@ & Dirección de correo electrónico del destinatario\\\hline
\ttfamily\itshape @@WebSite@@ & Página web del destinatario\\\hline\hline
\ttfamily @@FromSurname@@ & Apellido del remitente \\\hline
\ttfamily @@FromGivenName@@ & Nombre de pila del remitente \\\hline
\ttfamily @@FromTitle@@ & Tratamiento del remitente \\\hline
\ttfamily @@FromPosition@@ & Cargo del remitente \\\hline
\ttfamily @@FromDepartment@@ & Departamento del remitente\\\hline
\ttfamily @@FromCompanyName@@ & Nombre de la empresa del remitente (sin el departamento)\\\hline
\ttfamily @@FromStreet@@ & Nombre de la calle del remitente \\\hline
\ttfamily @@FromPlace@@ & Ubicación del remitente (sin la calle ni el código postal)\\\hline
\ttfamily @@FromZIPCode@@ & Código postal del remitente \\\hline
\ttfamily @@FromState@@ & Estado o provincia del remitente\\\hline
\ttfamily @@FromCountry@@ & País del remitente\\\hline
\ttfamily @@FromEMail@@ & Dirección de correo electrónico del remitente\\\hline
\ttfamily @@FromWebSite@@ & Página web del remitente\\\hline
\ttfamily @@TotalPageCount@@ & Número de páginas \textit{incluyendo} la página de portada \\\hline
\end{tabular}
\end{center}

El reemplazo se realiza a nivel de código fuente, por lo que esas palabras no se reconocerán en los cambios de formato dentro del mismo (por ejemplo, \texttt{@@sub\textit{ject@@}}).

\subsection{Me gustaba más el antiguo cuadro de diálogo. ¿Puedo volver a utilizarlo?}

Simplemente abra el cuadro de diálogo de Opciones y seleccione \texttt{Tradicional} en ``Estilo para el cuadro de diálogo de envío''.

\subsection{Quiero acceder a la agenda telefónica por medio de JDBC pero YajHFC no encuentra el controlador aunque especifique la ruta correcta para llamar a Java.}

Si utiliza el argumento \texttt{-jar}, Java ignora la ruta definida por el usuario.
Por tanto inicie YajHFC utilizando el siguiente comando (reemplace \texttt{/ruta/a/controladorbdd.jar} y \texttt{/ruta/a/yajhfc.jar} con sus respectivas rutas reales y nombre de archivo (por supuesto):
\begin{description}
\item [Linux/Unix:] \verb#java -classpath /ruta/a/controaldorbdd.jar:/ruta/a/yajhfc.jar yajhfc.Launcher#
\item [Windows:] \verb#java -classpath c:\ruta\a\controladorbdd.jar;c:\ruta\a\yajhfc.jar yajhfc.Launcher#
\end{description}

\subsection{¿Qué puedo poner como valor \texttt{igual a} en el cuadro de diálogo de filtros personalizados? }

Expresiones regulares. Puede encontrar una breve referencia sobre la sintaxis permitida en
\url{http://java.sun.com/j2se/1.5.0/docs/api/java/util/regex/Pattern.html}

Tenga en cuenta que las expresiones regulares no son lo mismo que los caracteres comodín:
Por ejemplo, para obtener el mismo efecto que el comodín \verb.*. tiene que utilizar \verb#.*#
y para simular el efecto de \verb#?# hay que utilizar \verb#.#.

\subsection{¿Qué argumentos de línea de comandos entiende YajHFC?}

\begin{verbatim}
Forma de uso:
java -jar yajhfc.jar [OPCIONES]... [ARCHIVOS PARA ENVIAR]...

Descripción de los argumentos:
-r, --recipient=DESTINATARIO            Especifica el destinatario al que se va
                                        a enviar un fax. Puede especificar el
                                        número de fax o bien información
                                        detallada de la página de portada
                                        (consulte la FAQ para obtener
                                        información del formato si elige esta
                                        última opción). Puede especificar
                                        varios --recipient en el caso de que
                                        haya múltiples destinatarios.
-C, --use-cover[=yes|no]                Utilizar una página de portada para
                                        enviar un fax.
-s, --subject=ASUNTO                    El asunto del fax para la página de
                                        portada.
    --comment=COMENTARIO                Comentario para la página de portada.
    --stdin                             Leer el archivo para enviar desde una
                                        entrada estándar.
-A, --admin                             Iniciar en modo administrador.
-d, --debug                             Salida con información de depuración.
-l, --logfile=ARCHIVO DE REGISTRO       El archivo de registro donde almacenar
                                        la información de depuración (si no se
                                        especifica, utilizar la salida estándar
                                        "stdout").
    --appendlogfile=ARCHIVO DE REGISTRO Añadir la información de depuración al
                                        archivo de registro especificado.
    --background                        Si no hay ninguna instancia en
                                        ejecución de YajHFC, lanzar una nueva
                                        instancia y finalizar esta instancia
                                        (después de remitir el fax para
                                        enviar).
    --noclose                           No cerrar YajHFC después de enviar el
                                        fax.
-T, --showtab=0|R|1|S|2|T               Define la pestaña para mostrar al
                                        inicio. Especifique 0 o R para la
                                        pestaña de "Recibidos", 1 o S para la
                                        pestaña de "Enviados" o 2 o T para la
                                        pestaña de "Transmisión".
    --loadplugin=ARCHIVO JAR            Especifica un archivo jar de un
                                        complemento de YajHFC para cargar.
    --loaddriver=ARCHIVO JAR            Especifica la ubicación de un archivo
                                        jar para el controlador JDBC para
                                        cargar.
    --no-plugins                        Desactiva la carga de los complementos
                                        desde el archivo plugins.lst.
    --no-gui                            Envía un fax utilizando una interfaz de
                                        usuario mínima.
    --no-check                          Elimina la comprobación de la versión
                                        de Java al inicio.
-c, --configdir=DIRECTORIO              Define la configuración del directorio
                                        para utilizar en lugar de ~/.yajhfc
-h, --help[=COLUMNAS]                   Muestra este texto.
\end{verbatim}

\subsection{¿Cómo puedo pasar la información de la página de portada utilizando el parámetro \texttt{-{-}recipient}?}

Desde la versión 0.4.0, es posible pasar esa información utilizando las parejas \texttt{nombre:valor}, separadas por punto y coma. Por ejemplo, para enviar un fax a ``John Doe'' en ``Ciudad de ejemplo'' con el número de fax 0123456, hay que utilizar la siguiente línea de comandos:

\texttt{java -jar yajhfc.jar \textit{[...]} -{-}recipient="givenname:John;surname:Doe;location:Ciudad de ejemplo;faxnumber:0123456" \textit{[...]}}

Se reconocen los siguientes nombres de campo:
\begin{center}
\begin{tabular}{|l|p{.7\textwidth}|}
\hline
\bfseries Nombre del campo & \bfseries Significado \\
\hline\hline
\ttfamily surname & Apellido del destinatario\\\hline
\ttfamily givenname & Nombre de pila del destinatario \\\hline
\ttfamily title & Tratamiento del destinatario \\\hline
\ttfamily position & Cargo del destinatario \\\hline
\ttfamily department & The recipient's department\\\hline
\ttfamily company & Nombre de la empresa del destinatario\\\hline
\ttfamily street & Nombre de la calle del destinatario \\\hline
\ttfamily location & Ubicación del destinatario\\\hline
\ttfamily zipcode & Código postal del destinatario \\\hline
\ttfamily state & Estado o provincia del destinatario\\\hline
\ttfamily country & País del destinatario\\\hline
\ttfamily email & Dirección de correo electrónico del destinatario\\\hline
\ttfamily faxnumber & Número de fax del destinatario \\\hline
\ttfamily voicenumber & Número de teléfono del destinatario \\\hline
\ttfamily website & Página web del destinatario\\\hline
\end{tabular}
\end{center}

\subsection{¿Qué significa la columna XYZ?}

Yo tampoco lo sé exactamente porque la descripción de las columnas ha sido copiada del manual de \verb#faxstat(1)# (JobFmt/RcvFmt), abreviada y traducida.

\subsection{¿Cómo se pueden especificar algunos ajustes predeterminados?}

Desde la versión 0.4.0, se cargan los siguientes archivos (en el caso de que existan) para obtener los ajustes que se han guardado:
\begin{enumerate}
 \item \texttt{[Directorio donde se encuentra el archivo yajhfc.jar]/settings.default}
 \item los ajustes de usuario desde \texttt{\{user.home\}\footnote{En Windows \texttt{user.home} normalmente está en \texttt{C:\textbackslash Documents and Settings\textbackslash USERNAME}.}/.yajhfc/settings} (si se especifica el parámetro \texttt{-{-}configdir=DIR}, se utiliza en su lugar \texttt{DIR/settings})
 \item \texttt{[Directorio donde se encuentra el archivo yajhfc.jar]/settings.override}
\end{enumerate}

Los ajustes que se cargan posteriormente desde archivos sobreescriben los ajustes especificados en los archivos que se cargan al principio, es decir, los de \texttt{settings.override} tienen preferencia sobre los que se encuentran en los ajustes del usuario y \texttt{settings.default}.
\medskip

Esta lógica se puede utilizar para especificar los ajustes predeterminados (por ejemplo, en un entorno de red): \\
Sólo hay que configurar una instalación de YajHFC al gusto y luego copiar los archivos \texttt{\{user.home\}/.yajhfc/settings} en el directorio donde se encuentra \texttt{yajhfc.jar} y renombrar el archivo como \texttt{settings.default}.
\medskip

Las redefiniciones se pueden especificar de forma similar. En este caso, se recomienda editar el archivo de ajustes (es un archivo de texto plano), y eliminar cualquier línea donde de especifiquen los ajustes que no se quieran sobreescribir (normalmente querrá eliminar al menos \texttt{user} y \texttt{pass-obfuscated} (el nombre del usuario y la contraseña que se utiliza para conectar al servidor HylaFAX), \texttt{FromName}, \texttt{*ColState} (el ancho de las columnas de las tablas), \texttt{*Bounds} (la posición de las diferentes ventanas) y \texttt{mainwinLastTab}).

Tenga en cuenta que el usuario aún puede definir estos ajustes con un valor diferente para una instancia de YajHFC que ya se encuentra en ejecuión. Sólo se restablecen a los valores redefinidos cuando se inicia YajHFC de nuevo (en otras palabras: el usuario puede definir estos ajustes pero no se guardan entre diferentes ejecuciones de YajHFC).


\subsection{¿Cómo se puede mostrar el directorio de HylaFAX \texttt{archive} en YajHFC?}

Desde la versión 0.4.0 se ha implementado el soporte para el directorio de HylaFAX \texttt{archive}.

No se puede acceder a este directorio utilizando la conexión "'normal"' de HylaFAX como en los otros directorios, ya que HylaFAX sólo permite obtener la lista de los subdirectorios que se encuentran bajo el directorio \texttt{archive}, pero no permite obtener ninguna información sobre los trabajos archivados (excepto el ID) o de los documentos adjuntados (si conoce alguna versión de HylaFAX donde se permita, por favor, hágamelo saber).

Por este motivo, se tiene que acceder al directorio utilizando otros métodos. Actualmente (0.4.0) YajHFC sólo permite el acceso utilizando el sistema de archivos. Esto significa que tiene que exportar el directorio \texttt{archive} del servidor por medio de Samba, NFS o cualquier otro sistema de archivos de red, montarlo en el cliente (si usa Unix / Linux; en Windows puede usar las rutas UNC) e indicar a YajHFC donde puede encontrar el directorio \texttt{archive} desde el menú de Opciones.

Una vez hecho eso, la tabla del archivador debería funcionar como el resto de tablas.

\subsection{¿Qué hacen las diferentes opciones que se encuentran en \texttt{Rutas y Visores ->\ Ajustes para la visualización y el envío} (versiones 0.4 y superiores)?}

Para los impacientes: Los ajustes recomendados son (no son los predeterminados porque necesitan \texttt{gs} y \texttt{tiff2pdf}):
\begin{itemize}
 \item \textbf{Formato:} PDF o TIFF
 \item \textbf{Enviar múltiples archivos como:} Archivo único excepto para la página de portada
 \item \textbf{Mostrar como archivo único:} Sí
 \item \textbf{Visualizar/enviar siempre con este formato:} Sí
\end{itemize}

\subsubsection{Formato para visualizar/enviar}

El formato en el cual se convertirán los documentos en el caso de que sea necesario. Por lo general, PDF y TIFF proporcionan aquí mejores resultados que PostScript (ya que este último utiliza el dispositivo \texttt{pswrite} de GhostScript).

\subsubsection{Enviar múltiples archivos como}

{\parindent0pt
\textbf{Múltiples archivos:}\\
El mismo comportamiento que las versiones anteriores a la 0.4.0. Si adjunta varios documentos en un sólo trabajo de fax, estos documentos se convierten a PS o PDF, pero se mantienen como archivos separados (por ejemplo, si se envía un fax con con los archivos \texttt{documento.ps} e \texttt{imagen.jpg}, se envían dos archivos por separado)
\medskip

\textbf{Archivo único excepto para la página de portada:}\\
Se crea un archivo único para el fax completo pero la página de portada se mantiene como un archivo separado (por ejemplo, si se envía un fax con \texttt{documento.ps} e \texttt{imagen.jpg}, se genera un archivo único en formato PDF/PS/TIFF y se envía).\\
\textit{Ventaja:} Se reduce la resolución a 196ppp (\textrightarrow archivos más pequeños para subir) y se puede utilizar un archivo de documento único al enviar un fax a varios destinos.
\medskip

\textbf{Fax completo como archivo único:}\\
Se crea un archivo para el fax completo incluyendo la página de la portada. Si el fax no tiene página de portada, el comportamiento es idéntico al caso anterior.\\
\textit{Ventaja:} No es necesario relalizar conversiones en el cliente cuando se visualiza el fax enviado.\\
\textit{Desventaja:} Cuando se envía un fax a varios destinatarios se tiene que crear un archivo para cada destinatario y se tiene que enviar.
}

\subsubsection{Visualizar los faxes como un archivo único}
Si esta opción está activada y un fax del servidor consiste en varios archivos, se crea un archivo único (en el cliente) para visualizarlo.


\subsubsection{Visualizar/enviar faxes en este formato}
Esta opción modifica el comportamiento de ``Enviar múltiples archivos como'' y ``Visualizar los faxes como un archivo único''.\\
Cuando está opción se encuentra \textit{desactivada}, sólo se convierten los faxes cuando están formados por múltiples archivos. Si se trata solamente de un único archivo, el formato se mantiene como está.\\
Cuando esta opción está \textit{activada}, un fax que consiste en un único archivo también se convierte cuando ese archivo único tiene un formato diferente del seleccionado en ``Formato para visualizar/enviar''.\\
\textit{Ventaja:} Sólo se utiliza un visor para los faxes enviados y recibidos (por ejemplo, para ver los faxes recibidos en formato PDF).\\
\textit{Desventaja:} Normalmente hacen falta más conversiones entre formatos en el cliente.

\subsection{¿Qué caracteres se reconocen como formato para la fecha/hora?}

Para la fecha se utiliza el formato de Java \texttt{SimpleDateFormat}. Puede encontrar la descripción de los caracteres reconocidos en \url{http://java.sun.com/j2se/1.5.0/docs/api/java/text/SimpleDateFormat.html}.

\section{Problemas/Errores conocidos}

\subsection{¡He creado un documento / página de portada en HTML pero el formato en YajHFC se ve mal!}

YajHFC utiliza el soporta HTML que está integrado en Java (\texttt{HTMLEditorKit} / \texttt{HTMLDocument}) para convertir el HTML en PostScript. Es un soporte muy limitado ya que sólo soporta la especificación HTML 3.2.\\
Esto significa que los diseños complejos seguramente no se rendericen correctamente en YajHFC.
Para lograr el diseño que quieres, tienes básicamente las siguientes alternativas:

\begin{itemize}
 \item Utilizar el método de ``prueba \& error'' hasta que el diseño quede bien (el botón de previsualización en el cuadro de diálogo de envío mostrará el diseño del HTML convertido).
 \item Utilizar un editor HTML como Ekit (\url{http://www.hexidec.com/ekit.php}) que también utiliza el soporte HTML de Java, por lo que el documento se renderizará de manera similar en YajHFC.
 \item Utilizar otro formato para la página de portada (como XSL:FO u ODT con el complemento FOP).
\end{itemize}

\subsection{Cuando intento ver los faxes enviados siempre obtengo un mensaje de error 
   que dice \texttt{Formato de archivo PCL no soportado}, aunque todos los documentos 
   están el formato PostScript/PDF.}

Compruebe que la opción \texttt{Utilizar archivo PCL para la corrección de error} en el 
cuadro de Opciones está seleccionada e inténtelo de nuevo.
 
Algunas versiones de HylaFAX reportan 
un tipo de archivo "PCL" para todos los documentos asociados con un trabajo. 
Si esta opción está seleccionada, YajHFC intenta adivinar el tipo de archivo 
utilizando la extensión si se reporta PCL (lo cual generalmente funciona bastante bien).

\subsection{A menudo obtengo un error cuando envío dos o más faxes en una fila. ¿Qué puedo hacer?}

Parece que algunas versiones del servidor HylaFAX tienen problemas cuando se envía más de un fax en cada sesión.

Para solucionar este problema, hay que ir a la pestaña \texttt{Servidor} en el cuadro de diálogo Opciones, activar la casilla \texttt{Crear una nueva sesión por cada acción} y comprobar si el problema persiste.
Si hacer ésto \emph{no} soluciona el problema, por favor, envíame un correo electrónico con el informe del error.

\subsection{Cuando YajHFC se ejecuta en Windows algunas veces guarda su configuración en \texttt{C:\textbackslash .yajhfc} en lugar de \texttt{C:\textbackslash Documents and Settings\textbackslash USERNAME\textbackslash .yajhfc}}

De manera predeterminada, YajHFC guarda su configuración en el subdirectorio \texttt{.yajhfc} del directorio que devuelve
la propiedad del sistema Java \texttt{user.home}.
A veces, algunas versiones de Java parece que no establecen esta propiedad correctamente, lo que provoca dicho error.

Para solucionar ésto, se puede establecer esta propiedad explícitamente cuando se inicia YajHFC utilizando la línea de comandos de java \texttt{-D}, por ejemplo:\\
\texttt{java -Duser.home=\%USERPROFILE\% -jar "C:\textbackslash Archivos de Programa\textbackslash yajhfc.jar"}

\subsection{¡No se muestra el icono en la bandeja del sistema!}

Desde la versión 0.4.0 YajHFC admite la colocación de un icono en la bandeja del sistema que se mostrará cuando ejecute YajHFC bajo Java 1.6 (``Java 6'').
Si utiliza Java 1.5 (``Java 5''), no se admite el icono en la bandeja del sistema.

Por tanto, asegúrese de que tiene instalada Java 1.6. Si está absolutamente seguro que de tiene instalado Java 1.6 y aún así no se muestra el icono en la bandeja del sistema, por favor, envíeme un correo electrónico con el informe del fallo.


\section{Varios}

\subsection{¿Por qué se guardan las contraseñas en texto plano?}

Simplemente porque no hay otro método que sea mejor.


YajHFC puede codificar/"cifrar" las contraseñas antes de almacenarlas, 
pero si lo hace siempre es posible visualizar el código fuente para encontrarlas 
y descifrarlas (incluso aunque YajHFC fuera de código cerrado podrías reventarlas o experimentar un poco sobre cómo hacerlo).


El único método seguro requeriría introducir una contraseña maestra cada vez que se inicia YajHFC, pero en mi opinión no es mejor que introducir la contraseña real.

Debido a las diversas peticiones recibidas, a partir de la versión 0.4.0 (y superiores), las contraseñas están ofuscadas utilizando un algoritmo sencillo.
Sin embargo, lo mencionado anteriormente sigue siendo cierto, es decir, una vez que se lee el código fuente, las contraseñas se pueden descifrar fácilmente.

\subsection{¿Por qué escogiste ese nombre tan tonto?}

YajHFC empezó como "proyecto de prueba" para Java y la biblioteca \texttt{gnu.hylafax}
y por lo tanto no tuvo un nombre "bonito". Después de trabajar un poco sobre ello, me di cuenta de que se volvería útil, por lo que elegí darle un nombre.
Dado que estaba con la herramienta Yast de SuSE en ese momento y sabía 
que había / hay otros clientes Java para HylaFAX, lo llamé "\textbf{y}et \textbf{a}nother \textbf{J}ava \textbf{H}ylaFAX \textbf{c}lient" (en español, "otro cliente Java para HylaFAX").
 

\end{document}
