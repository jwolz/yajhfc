\documentclass[a4paper,10pt]{scrartcl}
\usepackage{ucs}
\usepackage[utf8x]{inputenc}
\usepackage{url}
\usepackage[left=3cm,right=3cm,top=2cm,bottom=2cm,nohead]{geometry}
\usepackage[spanish]{babel}
\usepackage[colorlinks=false,pdfborder=0 0 0]{hyperref}
\usepackage[T1]{fontenc}

%opening
\title{YajHFC FAQ}
\author{}
\date{}
\begin{document}

\maketitle

\tableofcontents

\section{Instalación}
\subsection{¿Dónde puedo conseguir un visor de faxes para Windows?}

Generalmente puedes utilizar cualquier programa que permita mostrar archivos TIFF, aunque algunos de ellos muestran los faxes en una resolución baja con una anchura (a la mitad) incorrecta.

Afortunadamente, las versiones recientes de Windows vienen con un programa 
que hace ésto correctamente:

\begin{description}
\item[Windows 95/98/ME/2000:]
Puedes utilizar el programa que se encuentra en Accesorios / Imagen. 
Para utilizarlo con YajHFC, utiliza el botón de exploración en \texttt{Línea de comandos 
para el visor de faxes} y selecciona la ubicación de kodakimg.exe en tu disco duro.

\item[Windows XP:]
Puedes utilizar el programa integrado "Visor de imágenes y faxes". 
Para utilizarlo, introduce el siguiente texto en "Línea de comandos para el visor de faxes":\\
      \verb#rundll32.exe shimgvw.dll,ImageView_Fullscreen %s#
 \end{description}

      
\subsection{¿Dónde puedo obtener un visor de faxes para Linux/*BSD/... ?}
\texttt{kfax} funciona muy bien, pero al igual que en Windows, puedes utilizar cualquier programa que pueda mostrar archivos TIFF, tan sólo busca en la base de paquetes de tu distribución (xloadimage \emph{no} funciona).


\subsection{¿Qué puedo utilizar como visor de faxes en Mac OS X?}
Sólo tienes que introducir \verb#open %s# en \texttt{Línea de comandos para el visor de faxes".}
Los faxes deberían abrirse con la aplicación predeterminada. 
Gracias a Scott Harris por esta ayuda.


\subsection{¿Dónde puedo obtener un visor PostScript?}
\begin{description}
\item[Windows:] Utiliza Ghostview disponible en
\url{http://www.cs.wisc.edu/~ghost/}\\
\textbf{Nota:} también es posible utilizar el programa Acrobat Distiller
(verificado sólo con la versión completa) que permite abrir los archivos PS
y visualizarlo al instante. Para ello hay que introducir la ruta al programa
en \texttt{Línea de comandos para el visor archivos PostScript},
generalmente:
\verb#"C:\Archivos de programa\Adobe\Acrobat 5.0\Distillr\Acrodist.exe" %s#

\item[Linux/*BSD/...:] Simplemente instala uno de los paquetes de visores
PostScript
(por ejemplo: \texttt{gv, kghostview, gnome-gv, ...})
\end{description}


\section{Utilización del programa}	

\subsection{¿Cómo puedo editar las plantillas PostScript de las portadas?}
Las plantillas tienen que estar en el mismo formato PostScript especial 
que utiliza el programa \texttt{faxcover} de HylaFAX. Revisa la siguiente ayuda 
para saber cómo crear / modificar este tipo de archivos: \\
\url{http://www.hylafax.org/HylaFAQ/Q202.html}\\
\url{http://www.hylafax.org/howto/tweaking.html}\\

Como alternativa, a partir de YajHFC 0.3.7, puedes utilizar páginas de portada en HTML o bien en formato XSL:FO u ODT (OpenDocument Text), mediante un complemento.

\subsection{¡He creado un documento / página de portada en HTML pero el formato en YajHFC se ve mal!}

YajHFC utiliza el soporta HTML que está integrado en Java (\texttt{HTMLEditorKit} / \texttt{HTMLDocument}) para convertir el HTML en PostScript. Es un soporte muy limitado ya que sólo soporta la especificación HTML 3.2.\\
Esto significa que los diseños complejos seguramente no se rendericen correctamente en YajHFC.
Para lograr el diseño que quieres, tienes básicamente las siguientes alternativas:

\begin{itemize}
 \item Utilizar el método de ``prueba \& error'' hasta que el diseño quede bien (el botón de previsualización en el cuadro de diálogo de envío mostrará el diseño del HTML convertido).
 \item Utilizar un editor HTML como Ekit (\url{http://www.hexidec.com/ekit.php}) que también utiliza el soporte HTML de Java, por lo que el documento se renderizará de manera similar en YajHFC.
 \item Utilizar otro formato para la página de portada (como XSL:FO u ODT con el complemento FOP).
\end{itemize}

\subsection{¿Qué campos se reconocen en una página de portada HTML?}

Las siguientes ``palabras'' se reemplazan (sin tener en cuenta mayúsculas / minúsculas) con sus valores correspondientes cuando se utiliza un archivo HTML como página de portada:

\begin{center}
\begin{tabular}{|l|l|}
\hline
\bfseries Palabra & \bfseries Significado \\
\hline\hline
\ttfamily @@Name@@ & Nombre del destinatario \\\hline
\ttfamily @@Location@@ & Ubicación del destinatario \\\hline
\ttfamily @@Company@@ & Empresa del destinatario \\\hline
\ttfamily @@Faxnumber@@ & Número de fax del destinatario \\\hline
\ttfamily @@Voicenumber@@ & Número de teléfono del destinatario \\\hline
\ttfamily @@FromName@@ & Nombre del emisor \\\hline
\ttfamily @@FromLocation@@ & Ubicación del emisor \\\hline
\ttfamily @@FromCompany@@ & Empresa del emisor \\\hline
\ttfamily @@FromFaxnumber@@ & Número de fax del emisor \\\hline
\ttfamily @@FromVoicenumber@@ & Número de teléfono del emisor \\\hline
\ttfamily @@FromEMail@@ & Dirección de correo electrónico del emisor \\\hline
\ttfamily @@Subject@@ & Asunto del fax \\\hline
\ttfamily @@Date@@ & Fecha actual \\\hline
\ttfamily @@PageCount@@ & Número de páginas sin incluir la página de portada \\\hline
\ttfamily @@Comments@@ & Comentarios para el fax \\\hline
\end{tabular}
\end{center}

El reemplazo se realiza a nivel de código fuente, por lo que esas palabras no se reconocerán en los cambios de formato dentro del mismo (por ejemplo, \texttt{@@sub\textit{ject@@}}).

\subsection{Me gustaba más el antiguo cuadro de diálogo. ¿Puedo volver a utilizarlo?}

Simplemente abre el cuadro de diálogo de Opciones y selecciona \texttt{Tradicional} en ``Estilo para el cuadro de diálogo de envío''.

\subsection{Quiero acceder a la agenda telefónica por medio de JDBC pero YajHFC no encuentra el controlador aunque especifique la ruta correcta para llamar a Java.}

Si utilizas el argumento \texttt{-jar}, Java ignora la ruta definida por el usuario.
Por tanto inicia YajHFC utilizando el siguiente comando (reemplaza \texttt{/ruta/a/controladorbdd.jar} y \texttt{/ruta/a/yajhfc.jar} con sus respectivas rutas reales y nombre de archivo (por supuesto):
\begin{description}
\item [Linux/Unix:] \verb#java -classpath /ruta/a/controaldorbdd.jar:/ruta/a/yajhfc.jar yajhfc.Launcher#
\item [Windows:] \verb#java -classpath c:\ruta\a\controladorbdd.jar;c:\ruta\a\yajhfc.jar yajhfc.Launcher#
\end{description}

\subsection{¿Qué puedo poner como valor \texttt{igual a} en el cuadro de diálogo de filtrado personalizado? }

Expresiones regulares. Puedes encontrar una breve referencia sobre la sintaxis permitida en
\url{http://java.sun.com/j2se/1.5.0/docs/api/java/util/regex/Pattern.html}

Ten en cuenta que las expresiones regulares no son lo mismo que los caracteres comodín:
Por ejemplo, para obtener el mismo efecto que el comodín \verb.*. tienes que utilizar \verb#.*#
y para simular el efecto de \verb#?# hay que utilizar \verb#.#.

\subsection{¿Qué argumentos de línea de comandos entiende YajHFC?}

\begin{verbatim}
Uso general:
java -jar yajhfc.jar [--help] [--debug] [--admin] [--background|--noclose]
         [--configdir=directory] [--loadplugin=filename] [--logfile=filename]
         [--showtab=0|R|1|S|2|T] [--recipient=...] [--stdin | filename ...]
Descripción de los argumentos:
filename     El nombre del archivo PostScript para enviar.
--stdin      Lee el archivo para enviar de una entrada estándar
--recipient  Especifica el número de teléfono del destinatario para enviar un fax.
             Es posible especificar varios valores para múltiples destinatarios.
--admin      Inicia en modo administrador
--debug      Salida de información de errores
--logfile    Archivo de registro donde registrar la información de depuración (si no se especifica, usar la salida estándar "stdout")
--background Si no hay ninguna instancia en ejecución, iniciar una nueva instancia
             y terminar (después de enviar el archivo)
--noclose    No cerrar YajHFC después de enviar el fax
--showtab    Establece la pestaña para mostrar al inicio. Especifica 
             0 o R para la pestaña "Recibidos", 1 o S para "Enviados" o
             2 o T para "Transmistiendo".
--loadplugin Especifica el archivo jar de un plugin de YajHFC para cargar
--no-plugins Descativa la carga de complementos desde el archivo plugin.lst.
--configdir  Establece un directorio de configuración para utilizar en lugar de ~/.yajhfc
--help       Muestra este texto
\end{verbatim}


\subsection{Cuando intento ver los faxes enviados siempre obtengo un mensaje de error 
   que dice \texttt{Formato de archivo PCL no soportado}, aunque todos los documentos 
   están el formato PostScript/PDF.}

Verifica que la opción \texttt{Utilizar archivo PCL para la corrección de error} en el 
cuadro de Opciones está seleccionada e inténtalo de nuevo.
 
Algunas versiones de HylaFAX reportan 
un tipo de archivo "PCL" para todos los documentos asociados con un trabajo. 
Si esta opción está seleccionada, YajHFC intenta adivinar el tipo de archivo 
utilizando la extensión si se reporta PCL (lo cual generalmente funciona bastante bien).

\subsection{Cuando YajHFC se ejecuta en Windows algunas veces guarda su configuración en \texttt{C:\textbackslash .yajhfc} en lugar de \texttt{C:\textbackslash Documents and Settings\textbackslash USERNAME\textbackslash .yajhfc}}

De manera predeterminada, YajHFC guarda su configuración en el subdirectorio \texttt{.yajhfc} del directorio que devuelve
la propiedad del sistema Java \texttt{user.home}.
A veces, algunas versiones de Java parece que no establecen esta propiedad correctamente, lo que provoca dicho error.

Para solucionar ésto, puedes establecer esta propiedad explícitamente cuando inicias YajHFC utilizando la línea de comandos de java \texttt{-D}, por ejemplo:\\
\texttt{java -Duser.home=\%USERPROFILE\% -jar "C:\textbackslash Archivos de Programa\textbackslash yajhfc.jar"}

\subsection{¿Qué significa la columna XYZ?}

Yo tampoco lo sé exactamente porque la descripción de las columnas ha sido copiada del manual de  \verb#faxstat(1)# (JobFmt/RcvFmt), abreviada y traducida

\section{Varios}

\subsection{¿Por qué se guardan las contraseñas en texto plano?}

Simplemente porque no hay otro método que sea mejor.


YajHFC puede codificar/"cifrar" las contraseñas antes de almacenarlas, 
pero si lo hace siempre es posible visualizar el código fuente para encontrarlas 
y descifrarlas (incluso aunque YajHFC fuera de código cerrado podrías reventarlas o experimentar un poco sobre cómo hacerlo).


El único método seguro requeriría introducir una contraseña maestra cada vez que se inicia YajHFC, pero en mi opinión no es mejor que introducir la contraseña real.


\subsection{¿Por qué escogiste ese nombre tan tonto?}

YajHFC empezó como "proyecto de prueba" para Java y la biblioteca \texttt{gnu.hylafax}
y por lo tanto no tuvo un nombre "bonito". Después de trabajar un poco sobre ello, me di cuenta de que se volvería útil, por lo que elegí darle un nombre.
Dado que estaba con la herramienta Yast de SuSE en ese momento y sabía 
que había / hay otros clientes Java para HylaFAX, lo llamé "\textbf{y}et \textbf{a}nother \textbf{J}ava \textbf{H}ylaFAX \textbf{c}lient" (en español, "otro cliente Java para HylaFAX").
 

\end{document}
