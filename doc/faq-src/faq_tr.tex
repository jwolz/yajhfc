\documentclass[a4paper,10pt]{scrartcl}
\usepackage{ucs}
\usepackage[utf8x]{inputenc}
\usepackage{url}
\usepackage[left=3cm,right=3cm,top=2cm,bottom=2cm,nohead]{geometry}
\usepackage[colorlinks=false,pdfborder=0 0 0]{hyperref}
\usepackage[T1]{fontenc}
\usepackage[turkish]{babel}
\usepackage{pslatex}

%opening
\title{YajHFC SSS}
\author{}
\date{}
\begin{document}

\sloppy
\maketitle

\tableofcontents

\section{Kurulum}
\subsection{Windows için faks görüntüleyici nereden temin edebilirim?}

Bazıları faksları düşük çözünürlükte yanlış boyutlarda (yarısı) gösterse de 
genellikle herhangi bir TIFF görüntüleyebilen yazılım kullanabilirsiniz.

Neyse ki tüm yeni windows sürümleri bu işi doğru yapan programla beraber geliyorlar.

\begin{description}
\item[Windows 95/98/ME/2000:]
Donatılar/Görüntüleme uygulamasını kullanabilirsiniz.
	YajHFC ile kullanmak için diskinizdeki \texttt{kodakimg.exe}'i seçmek için 
	\texttt{Faks görüntülüyeci için komut satırı}ndaki gözat düğmesini kullanın.\\
	Bu genellikle Windows dizininde (win 98/ME) \emph{veya} \\
\verb.Program Files\Windows NT\Accessories\ImageVue. dizininde bulunur. (win 2k). 
       
\item[Windows XP:]
Birleşik ``resim ve faks görünütleyici'' kullanabilirsiniz
	Kullanmak için aşağıdaki metini \texttt{Faks görüntüleyici için komut satırı} kutusuna girin:\\
	\verb#rundll32.exe shimgvw.dll,ImageView_Fullscreen %s#
 \end{description}

      
\subsection{Linux/*BSD/... için faks görünütüleyici nereden temin edebilirim?}
\texttt{kfax} bende gayet iyi çalışıyor fakat windowsdaki gibi TIFF dosyalarını görüntüleyen
her hangi bir yazılım kullanabilirsiniz, bu yazılımları dağıtımınızın paket yöneticisinden araştırabilirsiniz.


\subsection{Mac OS X'de faks görüntüleyici olarak ne kullanmalıyım?}
\texttt{Faks görüntüleyici için komut satırı} kutusuna sadece \verb.open %s. girmeniz yeterli.
Fakslar öntanımlı uyguluma ile açılacaktır.
Scott Harris'a bu çözüm için teşekkürler.

\subsection{PostScript görüntüleyici nerden bulabilirim?}
\begin{description}
\item[Linux/*BSD/...:] Sadece PostScript görüntüleyici pkaetlerden birini kurun.
		(örn. \texttt{gv, kghostview, gnome-gv, ...})


\end{description}

\subsection{Where can I get GhostScript from?}
\begin{description}
\item[Windows:] Download it from \url{http://www.cs.wisc.edu/~ghost/}
\item[Linux/*BSD/...:] Install the GhostScript package for your distribution (on most installations this package will already be installed; if not: the package name usually begins with \texttt{ghostscript} or \texttt{gs})
\end{description}

\subsection{Where can I get TIFF2PDF from?}
\begin{description}
\item[Windows:] Download it from \url{http://gnuwin32.sourceforge.net/downlinks/tiff.php}\\ If this link does not work, please see \url{http://gnuwin32.sourceforge.net/packages/tiff.htm} or \url{http://www.libtiff.org/}.
\item[Linux/*BSD/...:] Install the libtiff tools package for your distribution. Usually this package will have the word \texttt{tiff} in its name (on Debian/Ubuntu it is called \texttt{libtiff-tools} and on SUSE \texttt{tiff}).
\end{description}

\section{Program kullanımı}

\subsection{PostScript kapak sayfası şablonunu nasıl düzenleyebilirim?}
O şablonlar HylaFAX \texttt{faxcover} programının kullandığı aynı özel PostScript biçimini kullanmalı.
Bu tür dosyların nasıl oluşturulacağı ve düzenleneceğine dair ipucları için aşağıdaki safyalara bakabilirsiniz:


Seçenek olarak, YajHFC 0.3.7 sürümünden itibaren, kapak sayfalarını, HTML yada bir eklenti ile XSL:FO veya ODT (OpenDocument Text) biçiminde kullanabilirsiniz.
Alternatively, starting with YajHFC 0.3.7, you can use Cover pages in HTML or, with a plugin, in XSL:FO or ODT (OpenDocument Text) format.

 \subsection{Bir HTML kapak sayfasında hangi alanlar tanınır}
Bir HTML kapak sayfası kullanılırken, aşağıdaki ``kelimeler'' karşılığı olan değerlerle değiştirilir (büyük/küçük harf duyarlı):

\begin{center}
\begin{tabular}{|l|l|}
\hline
\bfseries Kelime & \bfseries Anlamı \\
\hline\hline
\ttfamily @@Name@@ & Alıcı adı \\\hline
\ttfamily @@Location@@ & Alıcı yeri (adresi) \\\hline
\ttfamily @@Company@@ & Alıcı şirketi \\\hline
\ttfamily @@Faxnumber@@ & Alıcı faks numarası \\\hline
\ttfamily @@Voicenumber@@ & Alıcı telefon numarası \\\hline
\ttfamily @@FromName@@ & Gönderen adı \\\hline
\ttfamily @@FromLocation@@ & Gönderen yeri (adresi) \\\hline
\ttfamily @@FromCompany@@ & Gönderen şirketi \\\hline
\ttfamily @@FromFaxnumber@@ & Gönderen faks numarası \\\hline
\ttfamily @@FromVoicenumber@@ & Gönderen telefon numarası \\\hline
\ttfamily @@FromEMail@@ & Gönderen email adresi \\\hline
\ttfamily @@Subject@@ & Faksın konusu \\\hline
\ttfamily @@Date@@ & Bugünün tarihi \\\hline
\ttfamily @@PageCount@@ & Kapak sayfası hariç faks sayfa adedi \\\hline
\ttfamily @@Comments@@ & Bu faks için girilen yorumlar \\\hline
\end{tabular}
\end{center}

Starting with version 0.4.0 the following additional fields are available (note that the \textit{italic} fields will be empty when you use resend fax):
\begin{center}
\begin{tabular}{|l|p{.7\textwidth}|}
\hline
\bfseries Word & \bfseries Meaning \\
\hline\hline
\ttfamily @@Surname@@ & The recipient's surname (resent faxes: the same as \texttt{@@Name@@}) \\\hline
\ttfamily\itshape @@GivenName@@ & The recipient's given name \\\hline
\ttfamily\itshape @@Title@@ & The recipient's title \\\hline
\ttfamily\itshape @@Position@@ & The recipient's position \\\hline
\ttfamily\itshape @@Department@@ & The recipient's department\\\hline
\ttfamily @@CompanyName@@ & The recipient's company name (without department) (resent faxes: the same as \texttt{@@Company@@})\\\hline
\ttfamily\itshape @@Street@@ & The recipient's street name \\\hline
\ttfamily @@Place@@ & The recipient's location (without street or ZIP code) (resent faxes: the same as \texttt{@@Location@@})\\\hline
\ttfamily\itshape @@ZIPCode@@ & The recipient's ZIP Code \\\hline
\ttfamily\itshape @@State@@ & The recipient's state\\\hline
\ttfamily\itshape @@Country@@ & The recipient's country\\\hline
\ttfamily\itshape @@EMail@@ & The recipient's e-mail address\\\hline
\ttfamily\itshape @@WebSite@@ & The recipient's website\\\hline\hline
\ttfamily @@FromSurname@@ & The sender's  surname \\\hline
\ttfamily @@FromGivenName@@ & The sender's  given name \\\hline
\ttfamily @@FromTitle@@ & The sender's  title \\\hline
\ttfamily @@FromPosition@@ & The sender's position \\\hline
\ttfamily @@FromDepartment@@ & The sender's  department\\\hline
\ttfamily @@FromCompanyName@@ & The sender's  company name (without department)\\\hline
\ttfamily @@FromStreet@@ & The sender's  street name \\\hline
\ttfamily @@FromPlace@@ & The sender's  location (without street or ZIP code)\\\hline
\ttfamily @@FromZIPCode@@ & The sender's  ZIP Code \\\hline
\ttfamily @@FromState@@ & The sender's  state\\\hline
\ttfamily @@FromCountry@@ & The sender's  country\\\hline
\ttfamily @@FromEMail@@ & The sender's  e-mail address\\\hline
\ttfamily @@FromWebSite@@ & The sender's website\\\hline
\ttfamily @@TotalPageCount@@ & The number of pages \textit{including} the cover page \\\hline
\end{tabular}
\end{center}

Bu değişimler kaynak kodu seviyesinde yapılır yani biçimleme, birisi içersinde yapılırsa bu kelimeler muhtemelen tanınmayacaktır. (örn. \texttt{@@sub\textit{ject@@}})

\subsection{Önceki iletiyi daha cok sevdim. Onu tekrar kullanabilir miyim?}

Seçenekler iletişimini açın ve ``Gönderme iletişim tipi'' olarak \texttt{Geleneksel} seçin.

\subsection{Telefon defterine JDBC üzerinden erişmek istiyorum ancak java çağrılırken doğru sınıf yolunu belirtmeme rağmen YajHFC sürücüyü bulamıyor.}

\texttt{-jar} komut satırı değişkenini kullanırsanız, java kullanıcı tanımlı sınıf yolunu gözardı eder.
Yani, lütfen YajHFC'yi aşağıdaki komutlarla başlatın (\texttt{/yol/dbdriver.jar} ve \texttt{/yol/yajhfc.jar}'ı elbette ilgili olan gerçek yol ve dosya adlarıyla değiştirin):

\begin{description}
\item [Linux/Unix:] \verb#java -classpath /yol/dbdriver.jar:/yol/yajhfc.jar yajhfc.Launcher#
\item [Windows:] \verb#java -classpath c:\yol\dbdriver.jar;c:\yol\yajhfc.jar yajhfc.Launcher#
\end{description}

\subsection{Özel filtre iletişiminde \texttt{matches} işleç için ne değer girebilirim?}

Düzenli İfadeler. Kabul edilen sözdizimi hakkında kısa bir başvuru kaynağı aşağıdadır:
\url{http://java.sun.com/j2se/1.5.0/docs/api/java/util/regex/Pattern.html}

Lütfen Düzenli İfadelerin joker karakterlerle aynı olmadığını dikkate alın:
Örneğin, \verb.*. joker karakterinin etkili olması için \verb#.*# kullanmak zorundasınız 
ve \verb#?# joker karakteri için \verb#.#.

\subsection{YajHFC hangi komut satırı değişkenlerini anlıyor?}

\begin{verbatim}
Genel Kullanım:
java -jar yajhfc.jar [--help] [--debug] [--admin] [--background|--noclose]
         [--configdir=directory] [--loadplugin=filename] [--logfile=filename]
         [--showtab=0|R|1|S|2|T] [--recipient=...] [--stdin | filename ...]
Değişken açıklaması:
Dosya adı	Gönderilecek PostScript dosya veya dosyalarının adı.
--stdin 	Standart girişten gönderilecek dosyaları okur
--recipient 	Faksı yollamak için alıcının telefon numarasını tanımlar.
		Çoklu alıcılar için çoklu değişken tanımlayabilirsiniz.
--admin		Yönetici kipinde başlatır
--debug		Hata ayıklama çıktısı verir
--logfile	Hata ayıklama çıktılarını kaydetmek için günlük dosyası (belirtilmezse, stdout kullanır)
--background	Eğer çalışan kopya yoksa, yeni bir kopya çalıştırır
		ve (dosya gönderilten sonra) sonlandırır.
--noclose	Faks gönderildikten sonra YajHFC'yi kapatmaz
--showtab	Açılışta gösterilecek sekmeleri ayarlar. "Gelen" için 0 yada R, 
		"Giden" için 1 yada S, "Gönderiliyor" sekmesi için 2 yada T tanımlayın.
--loadplugin	Yüklenecek YajHFC jar eklenti dosyasını tanımlar.
--no-plugins	plugin.lst dosyasından eklenti yüklenmesini devre dışı bırakır.
--configdir	~/.yajhfc yerine kullanılacak ayar dizinini tanımlar
--help		Bu metini görüntüler
\end{verbatim}




\subsection{How can I give cover page information using the \texttt{-{-}recipient} switch?}

Starting with version 0.4.0, you can give that information using \texttt{name:value} pairs, separated by semicolons. For example, to send a fax to ``John Doe'' in ``Example town'' with the fax number 0123456, use the following command line:

\texttt{java -jar yajhfc.jar \textit{[...]} -{-}recipient="givenname:John;surname:Doe;location:Example Town;faxnumber:0123456" \textit{[...]}}

The following field names are recognized:
\begin{center}
\begin{tabular}{|l|p{.7\textwidth}|}
\hline
\bfseries Field name & \bfseries Meaning \\
\hline\hline
\ttfamily surname & The recipient's surname\\\hline
\ttfamily givenname & The recipient's given name \\\hline
\ttfamily title & The recipient's title \\\hline
\ttfamily position & The recipient's position \\\hline
\ttfamily department & The recipient's department\\\hline
\ttfamily company & The recipient's company name\\\hline
\ttfamily street & The recipient's street name \\\hline
\ttfamily location & The recipient's location\\\hline
\ttfamily zipcode & The recipient's ZIP Code \\\hline
\ttfamily state & The recipient's state\\\hline
\ttfamily country & The recipient's country\\\hline
\ttfamily email & The recipient's e-mail address\\\hline
\ttfamily faxnumber & The recipient's fax number \\\hline
\ttfamily voicenumber & The recipient's voice number \\\hline
\ttfamily website & The recipient's website\\\hline
\end{tabular}
\end{center}

\subsection{XYZ sütunun anlamı nedir?}

Ben de tam olarak bilmiyorum çünkü sütun açıklamarı 
tamamen \verb.faxstat(1). kullanma klavuzundan (JobFmt/RcvFmt) kopyalanmıştır
ve kısaltılmıştır/çevirilmiştir. 

\subsection{How can I specify some default settings?}

Starting with version 0.4.0 the following files (if they exist) are loaded in order to retrieve the saved settings:
\begin{enumerate}
 \item \texttt{[Directory where yajhfc.jar resides]/settings.default}
 \item the user settings from \texttt{\{user.home\}\footnote{In Windows \texttt{user.home} usually is \texttt{C:\textbackslash Documents and Settings\textbackslash USERNAME}.}/.yajhfc/settings} (if you specified \texttt{-{-}configdir=DIR}, \texttt{DIR/settings} is used instead)
 \item \texttt{[Directory where yajhfc.jar resides]/settings.override}
\end{enumerate}

Settings from files loaded later override the settings specified in the files loaded earlier, i.e. the settings from \texttt{settings.override} take precedence over those in the user settings and \texttt{settings.default}.
\medskip

This logic can be used to specify default settings (e.g. in a networked environment): \\
Simply configure one YajHFC installation as you like, then copy \texttt{\{user.home\}/.yajhfc/settings} to the directory where \texttt{yajhfc.jar} is and rename it to \texttt{settings.default}.
\medskip

Overrides can be specified in a similar fashion. In this case, you are encouraged to edit the settings file (it is a plain text file), and remove any lines that specify settings you wish not to override (usually you will want to remove at least \texttt{user} and \texttt{pass-obfuscated} (user name and password used to connect to the HylaFAX server), \texttt{FromName}, \texttt{*ColState} (the widths of the table columns), \texttt{*Bounds} (the position of the different windows) and \texttt{mainwinLastTab}).

Note, however that the user can still set these settings to a different value for a running YajHFC instance. They are reset to the override values only when YajHFC is started again (in other words: the user can set these settings, but they are not saved between different runs of YajHFC).


\subsection{How can I display the HylaFAX \texttt{archive} folder in YajHFC?}

With version 0.4.0 support for the HylaFAX \texttt{archive} folder has been implemented.

This directory cannot be accessed using the "`normal"' HylaFAX connection like the other folders, however, because HylaFAX does only allow to get the list of subfolders in the \texttt{archive} directory, but not to get any information (except the ID) about the archived jobs or the attached documents (if you know a HylaFAX version where this is different, please let me know).

For this reason the directory has to be accessed by other methods. Currently (0.4.0) YajHFC only supports access using the file system. This means you have to export the  \texttt{archive} directory on the server using Samba, NFS or any other network file system, mount it on the client (if you use Unix; on Windows, you can just use UNC paths) and tell YajHFC in the options under which path the \texttt{archive} directory can be found.

When you have done that, the archive table should work just like the other tables do.

\subsection{What do the different options under \texttt{Paths \& Viewers -> View and send settings} do (0.4 and up)?}

For the impatient: The recommended settings are (they are not the default, because they require \texttt{gs} and \texttt{tiff2pdf}):
\begin{itemize}
 \item \textbf{Format:} PDF or TIFF
 \item \textbf{Send multiple files as:} Single file except cover
 \item \textbf{View as single file:} Yes
 \item \textbf{Always view/send in this format:} Yes
\end{itemize}

\subsubsection{Format for viewing/sending}

The format to which the documents should be converted if necessary. Generally PDF and TIFF give better results than PostScript here (as the latter uses GhostScript's \texttt{pswrite} device).

\subsubsection{Send multiple files as}

{\parindent0pt
\textbf{Multiple files:}\\
Same behaviour as pre-0.4.0 versions. If you attach multiple documents to a single fax job, these documents are converted to PS or PDF, but kept as separate files (e.g. if you send a fax with files \texttt{doc.ps} and \texttt{picture.jpg}, two separate files are uploaded)
\medskip

\textbf{Single file except for cover:}\\
One file is created for the whole fax, but the cover page is kept as a separate file (e.g. if you send a fax with \texttt{doc.ps} and \texttt{picture.jpg}, a single PDF/PS/TIFF file is created and uploaded).\\
\textit{Advantage:} The resolution is reduced to 196dpi (-> smaller files/upload) and a single document file can be used when sending a fax to multiple destinations.
\medskip

\textbf{Complete fax as single file:}\\
One file is created for the whole fax including the cover page. If the fax has no cover this behaves identical to the above case.\\
\textit{Advantage:} No conversions on the client are necessary when viewing the sent fax.\\
\textit{Disadvantage:} When sending a fax to multiple recipients one file per recipient has to be created and uploaded.
}

\subsubsection{View faxes as single file}
If this option is checked and a fax on the server consists of multiple files, a single file is created (on the client) for viewing.


\subsubsection{View/send faxes always in this format}
This option modifies the behaviour of ``Send multiple files as'' and ``View faxes as single file''.\\
When this option is \textit{unchecked}, a fax is only converted if it consists of multiple files. If it consists of only a single file, the format is left as it is.\\
When this option is \textit{checked}, a fax consisting of a single file is also converted when that single file has a different format than the one selected at ``Format for sending/viewing''.\\
\textit{Advantage:} A single viewer is used for both sent and received faxes (e.g. to view received faxes as PDF).\\
\textit{Disadvantage:} Usually more format conversions are necessary on the client.

\subsection{Which characters are recognized as date/time format?}

The date is formatted using is a Java \texttt{SimpleDateFormat}. A description of the characters recognized can be found at \url{http://java.sun.com/j2se/1.5.0/docs/api/java/text/SimpleDateFormat.html}.

\section{Problems/Known bugs}

\subsection{Bir HTML belge kapak sayfası oluşturdum fakat YajHFC'de biçim yanlış görünüyor!}

YajHFC, HTML'yi PostScripte çevirmek için Java (\texttt{HTMLEditorKit} / \texttt{HTMLDocument}) içersindeki birleşik HTML desteğini kullanır. Bu destek oldukça kısıtlıdır ki özellikle yanlız HTML 3.2'yi destekler.\\
Bunun anlamı karmaşık düzenler genelde YajHFC içinde doğru işlenmez.
İstediğiniz düzeni başarmak için aşağıdaki seçenekler mevcut:

\begin{itemize}
 \item Düzen doğru görünene kadar deneme yanılma yöntemini kullanabilirsiniz (Gönder iletisindeki önizleme düğmesi dönüştürülmüş HTML düzenini gösterecektir.)
 \item Başka bir kapak sayfası biçimi kullanabilirsiniz (XSL:FO veya FOP eklentisi ile ODT gibi)
\end{itemize}

\subsection{Faksları görüntülemeye çalıştığımda, bütün belgeler PostScript/PDF olmasına rağmen daima 
   \texttt{PCL dosya biçimi desteklenmiyor} hatası alıyorum.}

Seçenekler iletişimindeki \texttt{PCL dosya türü hata düzeltme kullan} onay kutusunu işaretleyin ve tekrar deneyin.

Bazı HylaFAX sürümleri, hatalı olarak, bir işlem ile ilgili tüm belgeler için
``PCL'' dosya türü bildirir. Eğer bu onay kutusu işaretlenirse, YajHFC, 
PCL dosya türü bildirildiğinde dosya türünü tahmin etmeye calışır (genellikle gayet iyi çalışır). 


\subsection{Bir yada birden fazla faksı bir dizide gönderdiğimde sık sık hata alıyorum. Bununla ilgili ne yapabilirim?}

HylaFAX sunucuların bazı sürümlerinde oturum başına birden fazla faks gönderildiğinde bu sorun oluşuyor gibi görünüyor.

Bu sorunu çözmek için Seçenekler iletişimindeki \texttt{Sunucu} sekmesine gidin, \texttt{Tüm işlemler için yeni oturum aç} kutusunu işaterleyin ve sorunun devam edip etmediğini deneyin.
İşe yaramazsa, lütfen hatayı bana mail atın.

\subsection{Windows altında çalışan YajHFC bazen kendi ayarlarını \texttt{C:\textbackslash Documents and Settings\textbackslash KULLANICIADI\textbackslash .yajhfc} yerine \texttt{C:\textbackslash .yajhfc} içerisine kaydediyor}

Öntanımlı olarak YajHFC kendi ayarlarını Java sistem özellikleri \texttt{user.home} tarafından döndürülen dizinin 
\texttt{.yajhfc} altdizini içersine kaydeder.
Bazen bazı Java sürümlerinde bu özellik doğru ayarlanmamış gibi görünüyor.

Çözüm olarak, YajHFC'yi Java'nın \texttt{-D} komut satırı anahtarı kullanarak başlatırken, bu özelliği ayarlayabilirsiniz. Örneğin: \\
\texttt{java -Duser.home=\%USERPROFILE\% -jar "C:\textbackslash Program Files\textbackslash yajhfc.jar"}

\subsection{The tray icon is not shown!}

Starting with version 0.4.0 YajHFC supports a tray icon which will be shown when you run YajHFC under Java 1.6 (``Java 6'').
If you use Java 1.5 (``Java 5''), a tray icon is not supported.

So, please make sure you have Java 1.6 installed. If you are absolutely sure you have Java 1.6 installed and the tray icon is still not shown, please mail me a bug report.

\section{Diğer}

\subsection{Neden parolalar şifresiz olarak kaydediliyor?}

Çünkü daha iyi bir yöntem yok.

YajHFC bir şekilde kaydetmeden önce kodlayabilirdi/``şifreleyebilirdi'' fakat
bunu yapsaydım, herzaman şifreyi çözmek için kaynak koda bakabilirdiniz 
(YajHFC kaynak kodu kapalı olsa dahi hala kaynak koda dönüştürebilir 
yada bunu yapmak için denemeler yapabilirdiniz).

Tek güvenli yöntem, YajHFC'yi çalıştırdığınızda her zaman sizden ana parolayı 
istemesidir, fakat benim fikrime göre bu, ``gerçek'' parolayı girmenin ötesinde
bir gelişme sağlamayacaktır.

Because of many requests passwords are obfuscated using a simple algorithm in version 0.4.0 and up.
The statement above still holds true, however, i.e. once you read the source code, the password can be decrypted easily.

\subsection{Neden böyle aptal bir isim seçtiniz?}

YajHFC, java ve \texttt{gnu.hylafax} kütüphaneleri için bir deneme projesi olarak başladı
yani ``güzel'' bir isime sahip değildi. Bir süre üzerinde çalıştıktan sonra 
gerçekten kullanılabilir bir duruma geldiğini farkettim ve bu ismi seçtim. 
Çünkü o sıralarda aynı zamanda SuSe'nin yast'ı ile de oynuyordum ve bir çok java hylaFAX istemcisi 
olduğunu biliyordum. Kısaca şöyle adlandırdım:
``\textbf{y}et \textbf{a}nother \textbf{J}ava \textbf{H}ylaFAX \textbf{c}lient''

\end{document}
