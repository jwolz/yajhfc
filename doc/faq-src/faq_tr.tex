\documentclass[a4paper,10pt]{scrartcl}
\usepackage{ucs}
\usepackage[utf8x]{inputenc}
\usepackage{url}
\usepackage[left=3cm,right=3cm,top=2cm,bottom=2cm,nohead]{geometry}
\usepackage[colorlinks=false,pdfborder=0 0 0]{hyperref}
\usepackage[T1]{fontenc}
\usepackage[turkish]{babel}

%opening
\title{YajHFC SSS}
\author{}
\date{}
\begin{document}

\maketitle

\tableofcontents

\section{Kurulum}
\subsection{Windows için faks görüntüleyici nereden temin edebilirim?}

Bazıları faksları düşük çözünürlükte yanlış boyutlarda (yarısı) gösterse de 
genellikle herhangi bir TIFF görüntüleyebilen yazılım kullanabilirsiniz.

Neyse ki tüm yeni windows sürümleri bu işi doğru yapan programla beraber geliyorlar.

\begin{description}
\item[Windows 95/98/ME/2000:]
Donatılar/Görüntüleme uygulamasını kullanabilirsiniz.
	YajHFC ile kullanmak için diskinizdeki \texttt{kodakimg.exe}'i seçmek için 
	\texttt{Faks görüntülüyeci için komut satırı}ndaki gözat düğmesini kullanın.\\
	Bu genellikle Windows dizininde (win 98/ME) \emph{veya} \\
\verb.Program Files\Windows NT\Accessories\ImageVue. dizininde bulunur. (win 2k). 
       
\item[Windows XP:]
Birleşik ``resim ve faks görünütleyici'' kullanabilirsiniz
	Kullanmak için aşağıdaki metini \texttt{Faks görüntüleyici için komut satırı} kutusuna girin:\\
	\verb#rundll32.exe shimgvw.dll,ImageView_Fullscreen %s#
 \end{description}

      
\subsection{Linux/*BSD/... için faks görünütüleyici nereden temin edebilirim?}
\texttt{kfax} bende gayet iyi çalışıyor fakat windowsdaki gibi TIFF dosyalarını görüntüleyen
her hangi bir yazılım kullanabilirsiniz, bu yazılımları dağıtımınızın paket yöneticisinden araştırabilirsiniz.


\subsection{Mac OS X'de faks görüntüleyici olarak ne kullanmalıyım?}
\texttt{Faks görüntüleyici için komut satırı} kutusuna sadece \verb.open %s. girmeniz yeterli.
Fakslar öntanımlı uyguluma ile açılacaktır.
Scott Harris'a bu çözüm için teşekkürler.




\subsection{PostScript görüntüleyici nerden bulabilirim?}
\begin{description}
\item[Linux/*BSD/...:] Sadece PostScript görüntüleyici pkaetlerden birini kurun.
		(örn. \texttt{gv, kghostview, gnome-gv, ...})


\end{description}
		
\section{Program kullanımı}

\subsection{PostScript kapak sayfası şablonunu nasıl düzenleyebilirim?}
O şablonlar HylaFAX \texttt{faxcover} programının kullandığı aynı özel PostScript biçimini kullanmalı.
Bu tür dosyların nasıl oluşturulacağı ve düzenleneceğine dair ipucları için aşağıdaki safyalara bakabilirsiniz:


Seçenek olarak, YajHFC 0.3.7 sürümünden itibaren, kapak sayfalarını, HTML yada bir eklenti ile XSL:FO veya ODT (OpenDocument Text) biçiminde kullanabilirsiniz.
Alternatively, starting with YajHFC 0.3.7, you can use Cover pages in HTML or, with a plugin, in XSL:FO or ODT (OpenDocument Text) format.

\subsection{Bir HTML belge kapak sayfası oluşturdum fakat YajHFC'de biçim yanlış görünüyor!}

YajHFC, HTML'yi PostScripte çevirmek için Java (\texttt{HTMLEditorKit} / \texttt{HTMLDocument}) içersindeki birleşik HTML desteğini kullanır. Bu destek oldukça kısıtlıdır ki özellikle yanlız HTML 3.2'yi destekler.\\
Bunun anlamı karmaşık düzenler genelde YajHFC içinde doğru işlenmez.
İstediğiniz düzeni başarmak için aşağıdaki seçenekler mevcut:

\begin{itemize}
 \item Düzen doğru görünene kadar deneme yanılma yöntemini kullanabilirsiniz (Gönder iletisindeki önizleme düğmesi dönüştürülmüş HTML düzenini gösterecektir.)
 \item Başka bir kapak sayfası biçimi kullanabilirsiniz (XSL:FO veya FOP eklentisi ile ODT gibi)
\end{itemize}

 \subsection{Bir HTML kapak sayfasında hangi alanlar tanınır}
Bir HTML kapak sayfası kullanılırken, aşağıdaki ``kelimeler'' karşılığı olan değerlerle değiştirilir (büyük/küçük harf duyarlı):

\begin{center}
\begin{tabular}{|l|l|}
\hline
\bfseries Kelime & \bfseries Anlamı \\
\hline\hline
\ttfamily @@Name@@ & Alıcı adı \\\hline
\ttfamily @@Location@@ & Alıcı yeri (adresi) \\\hline
\ttfamily @@Company@@ & Alıcı şirketi \\\hline
\ttfamily @@Faxnumber@@ & Alıcı faks numarası \\\hline
\ttfamily @@Voicenumber@@ & Alıcı telefon numarası \\\hline
\ttfamily @@FromName@@ & Gönderen adı \\\hline
\ttfamily @@FromLocation@@ & Gönderen yeri (adresi) \\\hline
\ttfamily @@FromCompany@@ & Gönderen şirketi \\\hline
\ttfamily @@FromFaxnumber@@ & Gönderen faks numarası \\\hline
\ttfamily @@FromVoicenumber@@ & Gönderen telefon numarası \\\hline
\ttfamily @@FromEMail@@ & Gönderen email adresi \\\hline
\ttfamily @@Subject@@ & Faksın konusu \\\hline
\ttfamily @@Date@@ & Bugünün tarihi \\\hline
\ttfamily @@PageCount@@ & Kapak sayfası hariç faks sayfa adedi \\\hline
\ttfamily @@Comments@@ & Bu faks için girilen yorumlar \\\hline
\end{tabular}
\end{center}

Bu değişimler kaynak kodu seviyesinde yapılır yani biçimleme, birisi içersinde yapılırsa bu kelimeler muhtemelen tanınmayacaktır. (örn. \texttt{@@sub\textit{ject@@}})

\subsection{Önceki iletiyi daha cok sevdim. Onu tekrar kullanabilir miyim?}

Seçenekler iletişimini açın ve ``Gönderme iletişim tipi'' olarak \texttt{Geleneksel} seçin.

\subsection{Telefon defterine JDBC üzerinden erişmek istiyorum ancak java çağrılırken doğru sınıf yolunu belirtmeme rağmen YajHFC sürücüyü bulamıyor.}

\texttt{-jar} komut satırı değişkenini kullanırsanız, java kullanıcı tanımlı sınıf yolunu gözardı eder.
Yani, lütfen YajHFC'yi aşağıdaki komutlarla başlatın (\texttt{/yol/dbdriver.jar} ve \texttt{/yol/yajhfc.jar}'ı elbette ilgili olan gerçek yol ve dosya adlarıyla değiştirin):

\begin{description}
\item [Linux/Unix:] \verb#java -classpath /yol/dbdriver.jar:/yol/yajhfc.jar yajhfc.Launcher#
\item [Windows:] \verb#java -classpath c:\yol\dbdriver.jar;c:\yol\yajhfc.jar yajhfc.Launcher#
\end{description}

\subsection{Özel filtre iletişiminde \texttt{matches} işleç için ne değer girebilirim?}

Düzenli İfadeler. Kabul edilen sözdizimi hakkında kısa bir başvuru kaynağı aşağıdadır:
\url{http://java.sun.com/j2se/1.5.0/docs/api/java/util/regex/Pattern.html}

Lütfen Düzenli İfadelerin joker karakterlerle aynı olmadığını dikkate alın:
Örneğin, \verb.*. joker karakterinin etkili olması için \verb#.*# kullanmak zorundasınız 
ve \verb#?# joker karakteri için \verb#.#.

\subsection{YajHFC hangi komut satırı değişkenlerini anlıyor?}

\begin{verbatim}
Genel Kullanım:
java -jar yajhfc.jar [--help] [--debug] [--admin] [--background|--noclose]
         [--configdir=directory] [--loadplugin=filename] [--logfile=filename]
         [--showtab=0|R|1|S|2|T] [--recipient=...] [--stdin | filename ...]
Değişken açıklaması:
Dosya adı	Gönderilecek PostScript dosya veya dosyalarının adı.
--stdin 	Standart girişten gönderilecek dosyaları okur
--recipient 	Faksı yollamak için alıcının telefon numarasını tanımlar.
		Çoklu alıcılar için çoklu değişken tanımlayabilirsiniz.
--admin		Yönetici kipinde başlatır
--debug		Hata ayıklama çıktısı verir
--logfile	Hata ayıklama çıktılarını kaydetmek için günlük dosyası (belirtilmezse, stdout kullanır)
--background	Eğer çalışan kopya yoksa, yeni bir kopya çalıştırır
		ve (dosya gönderilten sonra) sonlandırır.
--noclose	Faks gönderildikten sonra YajHFC'yi kapatmaz
--showtab	Açılışta gösterilecek sekmeleri ayarlar. "Gelen" için 0 yada R, 
		"Giden" için 1 yada S, "Gönderiliyor" sekmesi için 2 yada T tanımlayın.
--loadplugin	Yüklenecek YajHFC jar eklenti dosyasını tanımlar.
--no-plugins	plugin.lst dosyasından eklenti yüklenmesini devre dışı bırakır.
--configdir	~/.yajhfc yerine kullanılacak ayar dizinini tanımlar
--help		Bu metini görüntüler
\end{verbatim}


\subsection{Faksları görüntülemeye çalıştığımda, bütün belgeler PostScript/PDF olmasına rağmen daima 
   \texttt{PCL dosya biçimi desteklenmiyor} hatası alıyorum.}

Seçenekler iletişimindeki \texttt{PCL dosya türü hata düzeltme kullan} onay kutusunu işaretleyin ve tekrar deneyin.

Bazı HylaFAX sürümleri, hatalı olarak, bir işlem ile ilgili tüm belgeler için
``PCL'' dosya türü bildirir. Eğer bu onay kutusu işaretlenirse, YajHFC, 
PCL dosya türü bildirildiğinde dosya türünü tahmin etmeye calışır (genellikle gayet iyi çalışır). 


\subsection{Bir yada birden fazla faksı bir dizide gönderdiğimde sık sık hata alıyorum. Bununla ilgili ne yapabilirim?}

HylaFAX sunucuların bazı sürümlerinde oturum başına birden fazla faks gönderildiğinde bu sorun oluşuyor gibi görünüyor.

Bu sorunu çözmek için Seçenekler iletişimindeki \texttt{Sunucu} sekmesine gidin, \texttt{Tüm işlemler için yeni oturum aç} kutusunu işaterleyin ve sorunun devam edip etmediğini deneyin.
İşe yaramazsa, lütfen hatayı bana mail atın.

\subsection{Windows altında çalışan YajHFC bazen kendi ayarlarını \texttt{C:\textbackslash Documents and Settings\textbackslash KULLANICIADI\textbackslash .yajhfc} yerine \texttt{C:\textbackslash .yajhfc} içerisine kaydediyor}

Öntanımlı olarak YajHFC kendi ayarlarını Java sistem özellikleri \texttt{user.home} tarafından döndürülen dizinin 
\texttt{.yajhfc} altdizini içersine kaydeder.
Bazen bazı Java sürümlerinde bu özellik doğru ayarlanmamış gibi görünüyor.

Çözüm olarak, YajHFC'yi Java'nın \texttt{-D} komut satırı anahtarı kullanarak başlatırken, bu özelliği ayarlayabilirsiniz. Örneğin: \\
\texttt{java -Duser.home=\%USERPROFILE\% -jar "C:\textbackslash Program Files\textbackslash yajhfc.jar"}

\subsection{XYZ sütunun anlamı nedir?}

Ben de tam olarak bilmiyorum çünkü sütun açıklamarı 
tamamen \verb.faxstat(1). kullanma klavuzundan (JobFmt/RcvFmt) kopyalanmıştır
ve kısaltılmıştır/çevirilmiştir. 

\section{Diğer}

\subsection{Neden parolalar şifresiz olarak kaydediliyor?}

Çünkü daha iyi bir yöntem yok.

YajHFC bir şekilde kaydetmeden önce kodlayabilirdi/``şifreleyebilirdi'' fakat
bunu yapsaydım, herzaman şifreyi çözmek için kaynak koda bakabilirdiniz 
(YajHFC kaynak kodu kapalı olsa dahi hala kaynak koda dönüştürebilir 
yada bunu yapmak için denemeler yapabilirdiniz).

Tek güvenli yöntem, YajHFC'yi çalıştırdığınızda her zaman sizden ana parolayı 
istemesidir, fakat benim fikrime göre bu, ``gerçek'' parolayı girmenin ötesinde
bir gelişme sağlamayacaktır.


subsection{Neden böyle aptal bir isim seçtiniz?}

YajHFC, java ve \texttt{gnu.hylafax} kütüphaneleri için bir deneme projesi olarak başladı
yani ``güzel'' bir isime sahip değildi. Bir süre üzerinde çalıştıktan sonra 
gerçekten kullanılabilir bir duruma geldiğini farkettim ve bu ismi seçtim. 
Çünkü o sıralarda aynı zamanda SuSe'nin yast'ı ile de oynuyordum ve bir çok java hylaFAX istemcisi 
olduğunu biliyordum. Kısaca şöyle adlandırdım:
``\textbf{y}et \textbf{a}nother \textbf{J}ava \textbf{H}ylaFAX \textbf{c}lient''

\end{document}
