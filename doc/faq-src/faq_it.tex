\documentclass[a4paper,10pt]{scrartcl}
\usepackage[utf8x]{inputenc}
\usepackage{url}
\usepackage[left=3cm,right=3cm,top=2cm,bottom=2cm,nohead]{geometry}
\usepackage[colorlinks=false,pdfborder=0 0 0]{hyperref}
\usepackage[T1]{fontenc}
\usepackage{pslatex}
\usepackage[italian]{babel}

%opening
\title{YajHFC FAQ}
\author{}
\date{}
\begin{document}
\sloppy

\maketitle

\tableofcontents

\section{Installazione}
\subsection{Dove trovo un visualizzatore fax per Windows?}

Normalmente puoi usare un qualsiasi programma che possa visualizzare i file TIFF, anche
se alcuni evidenziano i fax a bassa risoluzione con una altezza errata (metà).

Fortunatamente, tutte le recenti versioni di Windows includono un programma che li
evidenzia in modo corretto:

\begin{description}
\item[Windows 95/98/ME/2000:]
 Puoi usare l'applicazione Accessori/Imaging.
      Per usarla con YajHFC, utilizza il pulsante 'sfoglia' della casella
      \texttt{Comando per il visualizzatore fax} per selezionare \texttt{kodakimg.exe}
      sul tuo hard drive.\\
      Questo file generalmente si trova o nella directory Windows (Win 98/ME) \emph{o} \\
      in \verb.Programmi\Windows NT\Accessori\ImageVue. (Win 2k).
       
\item[Windows XP/Vista:]
Puoi usare il ``visualizzatore immagini e fax'' integrato.
      Per utilizzarlo, immetti il seguente comando nella casella \texttt{Comando per il visualizzatore fax}:\\
      \verb#rundll32.exe shimgvw.dll,ImageView_Fullscreen %s#
 \end{description}

      
\subsection{Dove trovo un visualizzatore fax per Linux/*BSD/... ?}
\texttt{kfax} funzioni bene per me, ma così come per Windows, puoi usare qualsiasi programma che 
possa visualizzare file TIFF, basta cercare nel database dei pacchetti della tua distribuzione 
(in ogni caso, xloadimage \emph{non} mi funziona correttamente).


\subsection{Cosa dovrei utilizzare come visualizzatore fax su Mac OS X?}
E' sufficiente immettere \verb.open %s. nella casella \texttt{Comando per il visualizzatore fax}.
I fax ora dovrebbero aprirsi nell'applicazione predefinita.
Ringrazio Scott Harris per questa informazione.


\subsection{Dove trovo un visualizzatore PostScript?}
\begin{description}
\item[Windows:] Usa Ghostview disponibile da \url{http://www.cs.wisc.edu/~ghost/}
\item[Linux/*BSD/...:] E' sufficiente installare uno qualsiasi dei visualizzatori PostScript
                (e.g. \texttt{gv, kghostview, gnome-gv, ...})
\end{description}

\subsection{Dove trovo GhostScript?}
\begin{description}
\item[Windows:] Scaricalo da \url{http://www.cs.wisc.edu/~ghost/}
\item[Linux/*BSD/...:] Installa il pacchetto GhostScript per la tua distribuzione (nella maggior parte delle installazioni questo pacchetto sarà già installato; in caso contrario: il nome del pacchetto generalmente inizia con \texttt{ghostscript} o \texttt{gs})
\end{description}

\subsection{Dove trovo TIFF2PDF?}
\begin{description}
\item[Windows:] Scaricalo da \url{http://gnuwin32.sourceforge.net/downlinks/tiff.php}\\ Se questo link non funziona, prova \url{http://gnuwin32.sourceforge.net/packages/tiff.htm} o \url{http://www.libtiff.org/}.
\item[Linux/*BSD/...:] Installa il pacchetto ``libtiff tools'' per la tua distribuzione. Generalmente questo pacchetto avrà la parola \texttt{tiff} nel suo nome (su Debian/Ubuntu è chiamato \texttt{libtiff-tools} e su SUSE \texttt{tiff}).
\end{description}

\subsection{Come posso fornire al setup di Windows le versioni già scaricate di tiff2pdf and GhostScript?}

A partire dalla versione 0.4.2 il file di setup Windows opzionalmente scarica e installa tiff2pdf e GhostScript.
Prima di iniziare il download il programma di setup ricerca nei seguenti percorsi eventuali versioni già scaricate:
\begin{enumerate}
 \item La directory dove si trova \texttt{setup.exe}
 \item Il desktop dell'utente (dove i file sono salvati se si sceglie di salvare i file scaricati)
\end{enumerate}

Nota che i nomi dei file devono essere esattamente quelli dei file scaricati (cioè \texttt{gs864w32.exe} (32 bit) o \texttt{gs864w64.exe} (64 bit) e \texttt{tiff-3.8.2-1.exe}).

\section{Utilizzo del programma}	

\subsection{Come posso modificare i modelli di copertina fax PostScript?}
Quei modelli devono essere nello stesso formato PostScript speciale che il programma HylaFAX
\texttt{faxcover} usa. Vedi le pagine seguenti per suggerimenti su come creare/modificare
file di questo tipo: \\
\url{http://www.hylafax.org/HylaFAQ/Q202.html}\\
\url{http://www.hylafax.org/howto/tweaking.html}\\

Alternativamente, ad iniziare da YajHFC 0.3.7, puoi usare copertine in HTML o, con un plugin, in formato XSL:FO o ODT (OpenDocument Text).

\subsection{Quali campi sono riconosciuti in una copertina HTML?}

Le seguenti ``stringhe'' sono sostituite (senza verifica su maiuscole/minuscole) con i corrispondenti valori quando un file HTML è usato come copertina fax:

\begin{center}
\begin{tabular}{|l|l|}
\hline
\bfseries Stringa & \bfseries Significato \\
\hline\hline
\ttfamily @@Name@@ & Nome del destinatario \\\hline
\ttfamily @@Company@@ & Società del destinatario \\\hline
\ttfamily @@Location@@ & Indirizzo del destinatario \\\hline
\ttfamily @@Faxnumber@@ & Numero fax del destinatario \\\hline
\ttfamily @@Voicenumber@@ & Numero telefono del destinatario \\\hline
\ttfamily @@FromName@@ & Nome del mittente \\\hline
\ttfamily @@FromCompany@@ & Società del mittente \\\hline
\ttfamily @@FromLocation@@ & Indirizzo del mittente \\\hline
\ttfamily @@FromFaxnumber@@ & Numero fax del mittente \\\hline
\ttfamily @@FromVoicenumber@@ & Numero telefono del mittente \\\hline
\ttfamily @@FromEMail@@ & Indirizzo e-mail del mittente \\\hline
\ttfamily @@Subject@@ & Oggetto del fax \\\hline
\ttfamily @@Date@@ & Data odierna \\\hline
\ttfamily @@PageCount@@ & Numero di pagine \textit{esclusa} la copertina \\\hline
\ttfamily @@Comments@@ & I commenti/note immessi per questo fax \\\hline
\end{tabular}
\end{center}

A partire dalla versione 0.4.0 sono disponibili i seguenti campi addizionali (nota che i campi in \textit{corsivo} saranno vuoti quando utilizzi ``Reinvia fax''):
\begin{center}
\begin{tabular}{|l|p{.7\textwidth}|}
\hline
\bfseries Stringa & \bfseries Significato \\
\hline\hline
\ttfamily @@Surname@@ & Cognome del destinatario (fax reinviati: uguale a \texttt{@@Name@@}) \\\hline
\ttfamily\itshape @@GivenName@@ & Nome proprio del destinatario \\\hline
\ttfamily\itshape @@Title@@ & Titolo del destinatario \\\hline
\ttfamily\itshape @@Position@@ & Posizione del destinatario \\\hline
\ttfamily\itshape @@Department@@ & Dipartimento del destinatario\\\hline
\ttfamily @@CompanyName@@ & Nome della società del destinatario (senza dipartimento) (fax reinviati: uguale a \texttt{@@Company@@})\\\hline
\ttfamily\itshape @@Street@@ & Nome della via del destinatario \\\hline
\ttfamily @@Place@@ & Indirizzo del destinatario (senza via o C.A.P.) (fax reinviati: uguale a \texttt{@@Location@@})\\\hline
\ttfamily\itshape @@ZIPCode@@ & C.A.P. del destinatario \\\hline
\ttfamily\itshape @@State@@ & Stato/regione del destinatario\\\hline
\ttfamily\itshape @@Country@@ & Nazione del destinatario\\\hline
\ttfamily\itshape @@EMail@@ & Indirizzo e-mail del destinatario\\\hline
\ttfamily\itshape @@WebSite@@ & Sito Web del destinatario\\\hline\hline
\ttfamily @@FromSurname@@ & Cognome del mittente \\\hline
\ttfamily @@FromGivenName@@ & Nome proprio del mittente \\\hline
\ttfamily @@FromTitle@@ & Titolo del mittente \\\hline
\ttfamily @@FromPosition@@ & Posizione del mittente \\\hline
\ttfamily @@FromDepartment@@ & Dipartimento del mittente\\\hline
\ttfamily @@FromCompanyName@@ & Nome della società del mittente (senza dipartimento)\\\hline
\ttfamily @@FromStreet@@ & Nome della via del mittente \\\hline
\ttfamily @@FromPlace@@ & Indirizzo del mittente (senza via o C.A.P.)\\\hline
\ttfamily @@FromZIPCode@@ & C.A.P. del mittente \\\hline
\ttfamily @@FromState@@ & Stato/regione del mittente\\\hline
\ttfamily @@FromCountry@@ & Nazione del mittente\\\hline
\ttfamily @@FromEMail@@ & Indirizzo e-mail del mittente\\\hline
\ttfamily @@FromWebSite@@ & Sito Web del mittente\\\hline
\ttfamily @@TotalPageCount@@ & Numero di pagine \textit{inclusa} la copertina \\\hline
\ttfamily @@CCNameAndFax@@ & The names and fax numbers of the other recipients of this fax in the format \textit{Name1 <faxnumber1>; Name2 <faxnumber2>; ...} (versions > 0.4.4 only)\\\hline
\end{tabular}
\end{center}

Le sostituzioni sono fatte a livello di codice sorgente, quindi le stringhe non saranno riconosciute se la formattazione cambia all'interno di una di esse (es.: \texttt{@@sub\textit{ject@@}}).

A partire dalla versione 0.4.2 sono supportate alcune semplici istruzioni condizionali. Sono implementate usando commenti HTML, quindi è necessario assicurarsi che il testo incluso in una istruzione condizionale non contenga alcun commento.
\begin{center}
\begin{tabular}{|l|p{.5\textwidth}|}
\hline
\bfseries Stringa & \bfseries Significato \\
\hline\hline
\ttfamily @@IfSomeFilled:\textit{campo1,campo2,...}@@ & Include il testo che segue solo se almeno uno dei campi specificati è compilato (cioè ha un valore con lunghezza > 0).\\\hline
\ttfamily @@IfAllFilled:\textit{campo1,campo2,...}@@ & Include il testo che segue solo se tutti i campi specificati sono compilati (cioè hanno un valore con lunghezza > 0).\\\hline
\ttfamily @@IfSomeEmpty:\textit{campo1,campo2,...}@@ & Include il testo che segue solo se almeno uno dei campi specificati è vuoto (cioè ha un valore con lunghezza = 0).\\\hline
\ttfamily @@IfAllEmpty:\textit{campo1,campo2,...}@@ & Include il testo che segue solo se tutti i campi specificati sono vuoti (cioè hanno un valore con lunghezza = 0).\\\hline
\ttfamily @@Else@@ & Include il testo che segue solo se l'ultima Istruzione-\texttt{@@If...@@} non era vera.\\\hline
\ttfamily @@EndIf@@ & Segna la fine dell'area influenzata dall'ultima if. Ogni \texttt{@@If...@@} deve avere esattamente una corrispondente istruzione \texttt{@@EndIf@@}.\\\hline
\end{tabular}
\end{center}

\subsection{Dove posso trovare la vecchia copertina fax predefinita utilizzata prima della versione 0.4.2?}

Puoi trovarla nell'archivio ``Coverpage examples'' disponibile qui \url{ftp://ftp.berlios.de/pub/yajhfc/download/coverpages.zip}

\subsection{Preferivo la vecchia finestra di dialogo di invio fax. Posso riaverla?}

E' sufficiente aprire la finestra Opzioni e selezionare \texttt{Tradizionale} come ``Stile della finestra di invio''.

\subsection{Voglio accedere ad una rubrica via JDBC, ma YajHFC non trova il driver anche se io specifico un corretto 'path class' per l'invocazione di java.}

Se utilizzi il parametro di linea di comando \texttt{-jar}, java ignora il 'path class' specificato dall'utente.
Quindi avvia YajHFC utilizzando i comandi seguenti (sostituisci \texttt{/path/to/dbdriver.jar} e \texttt{/path/to/yajhfc.jar} con i rispettivi path e nomi file reali naturalmente):
\begin{description}
\item [Linux/Unix:] \verb#java -classpath /path/to/dbdriver.jar:/path/to/yajhfc.jar yajhfc.Launcher#
\item [Windows:] \verb#java -classpath c:\path\to\dbdriver.jar;c:\path\to\yajhfc.jar yajhfc.Launcher#
\end{description}

Nelle versioni recenti di YajHFC puoi anche semplicemente aggiungere il driver utilizzando il pannello \texttt{Plugins \& JDBC} nella finestra Opzioni.

\subsection{Cosa posso immettere come valore per l'operatore \texttt{match} nella finestra filtro personalizzato?}

Espressioni Regolari. Un breve elenco delle forme sintattiche accettate si trova:
\url{http://java.sun.com/j2se/1.5.0/docs/api/java/util/regex/Pattern.html}

Nota che le Espressioni Regolari non sono la stessa cosa delle 'wildcard': 
Per esempio, per ottenere l'effetto della wildcard \verb.*., devi utilizzare \verb#.*# e 
per l'effetto della wildcard \verb#?# usa \verb#.#.

\subsection{Quali parametri da riga di comando sono accettati da YajHFC?}

Output di \verb#java -jar yajhfc.jar --help# (versione 0.4.2):
\begin{verbatim}
Sintassi:
java -jar yajhfc.jar [OPZIONI]... [FILE DA INVIARE]...

Descrizione parametri:
-r, --recipient=DESTINATARIO    Specifica un destinatario a cui inviare il fax.
                                Si può specificare un numero fax o informazioni
                                dettagliate per la copertina (vedere la FAQ per
                                il formato nel secondo caso). Si può 
                                specificare --recipient più volte per 
                                destinatari multipli. 
-C, --use-cover[=yes|no]        Usa una copertina per l'invio fax. 
-s, --subject=OGGETTO           Oggetto fax per la copertina. 
    --comment=COMMENTO          Commento/nota per la copertina. 
    --stdin                     Legge il file da inviare dallo standard input. 
-A, --admin                     Avvia in modalità amministratore. 
-d, --debug                     Produce informazioni di debug. 
-l, --logfile=FILE_DI_LOG       Il file di log su cui scrivere le informazioni 
                                di debug (se non specificato, usa standard 
                                output). 
    --appendlogfile=FILE_DI_LOG Accoda le informazioni di debug al file di log 
                                specificato. 
    --background                Se non esiste già una istanza attiva di YajHFC,
                                lancia una nuova istanza e quindi la termina 
                                (dopo aver accodato il fax per l'invio). 
    --noclose                   Non chiude YajHFC dopo aver accodato il fax. 
    --no-wait                   Non attende che la finestra di invio fax sia 
                                chiusa dall'utente. Se YajHFC è invocato più 
                                volte prima che l'utente chiuda la finestra di 
                                invio fax, i documenti specificati sono 
                                aggiunti alla lista file in quella finestra. 
                                Questa opzione implica l'opzione --background. 
-T, --showtab=0|R|1|S|2|T       Seleziona il tipo di elenco fax da evidenziare 
                                all'avvio. Specificare 0 o R per "Ricevuti", 1 
                                o S per "Inviati", 2 o T per "In uscita". 
    --windowstate=N|M|I|T       Imposta lo stato iniziale della finestra 
                                principale a _N_ormale, _M_assimizzata, ridotta
                                a _I_cona (minimizzata) o minimizzata nell'area
                                di no_T_ifica. 
    --loadplugin=FILE_JAR       Specifica un file jar di un plugin YajHFC da 
                                caricare. 
    --loaddriver=FILE_JAR       Specifica il percorso di un file JAR di un 
                                driver JDBC da caricare. 
    --no-plugins                Disabilita il caricamento di plugin dal file 
                                plugin.lst. 
    --no-gui                    Invia un fax con solo una GUI (interfaccia 
                                grafica) minimale. 
    --no-check                  Sopprime il controllo della versione di Java 
                                all'avvio. 
-c, --configdir=DIRECTORY       Imposta la directory di configurazione da 
                                utilizzare al posto di ~/.yajhfc. 
-h, --help[=COLONNE]            Evidenzia questo testo (formattato per un 
                                numero COLONNE di colonne se impostato). 
\end{verbatim}

\subsection{Come posso dare le informazioni per la copertina utilizzando il parametro \texttt{-{-}recipient} ?}

A partire dalla versione 0.4.0, puoi dare queste informazioni utilizzando coppie \texttt{nome:valore}, separate da 'punto e virgola'. Per esempio, per inviare un fax a ``Mario Rossi'' in ``Città di esempio'' con numero fax 0123456, utilizza la linea di comando seguente:

\texttt{java -jar yajhfc.jar \textit{[...]} -{-}recipient="givenname:Mario;surname:Rossi;location:Città di esempio;faxnumber:0123456" \textit{[...]}}

I seguenti nomi campi sono riconosciuti:
\begin{center}
\begin{tabular}{|l|p{.7\textwidth}|}
\hline
\bfseries Nome campo & \bfseries Significato \\
\hline\hline
\ttfamily surname & Cognome del destinatario\\\hline
\ttfamily givenname & Nome proprio del destinatario \\\hline
\ttfamily title & Titolo del destinatario \\\hline
\ttfamily position & Posizione del destinatario \\\hline
\ttfamily department & Dipartimento del destinatario\\\hline
\ttfamily company & Nome della società del destinatario\\\hline
\ttfamily street & Nome della via del destinatario \\\hline
\ttfamily location & Indirizzo del destinatario\\\hline
\ttfamily zipcode & C.A.P. del destinatario \\\hline
\ttfamily state & Stato/regione del destinatario\\\hline
\ttfamily country & Nazione del destinatario\\\hline
\ttfamily email & Indirizzo e-mail del destinatario\\\hline
\ttfamily faxnumber & Numero fax del destinatario \\\hline
\ttfamily voicenumber & Numero telefono del destinatario \\\hline
\ttfamily website & Sito Web del destinatario\\\hline
\end{tabular}
\end{center}

A partire dalla versione 0.4.2 puoi anche specificare un file di testo di input contenente i destinatari utilizzando \texttt{-{-}recipient=@\textit{nomefile}} o il rispettivo pulsante nella finestra di invio fax.
Il file di testo deve contenere un destinatario per riga (un numero fax o un insieme di dati usando il formato di cui sopra).

\subsection{Quale è il significato della colonna XYZ?}

Molto probabilmente non lo so neanche io, perché le descrizioni delle colonne
sono semplicemente copiate dal manuale del comando \verb.faxstat(1). (JobFmt/RcvFmt) e
abbreviate/tradotte.

\subsection{Come posso specificare alcune impostazioni predefinite?}

A partire dalla versione 0.4.0 i seguenti file (se esistono) sono letti per caricare le impostazioni salvate:
\begin{enumerate}
 \item \texttt{[Directory dove si trova il file yajhfc.jar]/settings.default}
 \item i settaggi dell'utente da \texttt{\{user.home\}\footnote{In Windows \texttt{user.home} è generalmente \texttt{C:\textbackslash Documents and Settings\textbackslash USERNAME}.}/.yajhfc/settings} (se hai specificato \texttt{-{-}configdir=DIR}, viene usato \texttt{DIR/settings})
 \item \texttt{[Directory dove si trova il file yajhfc.jar]/settings.override}
\end{enumerate}

Le impostazioni presenti nei file caricati dopo hanno la precedenza su quelle specificate nei file caricati prima, cioè le impostazioni in \texttt{settings.override} hanno la precedenza su quelle dell'utente e su \texttt{settings.default}.
\medskip

Questa logica può essere utilizzata per specificare le impostazioni predefinite (es.: in un ambiente di rete): \\
Configura una installazione di YajHFC come preferisci, quindi copia \texttt{\{user.home\}/.yajhfc/settings} nella directory dove si trova il file \texttt{yajhfc.jar} e rinominalo \texttt{settings.default}.
\medskip

Le impostazioni forzate/obbligatorie possono essere fatte in un modo analogo. In questo caso, puoi modificare il file delle impostazioni (è un puro file di testo), e rimuovere ogni riga che specifichi una impostazione che non vuoi forzare (generalmente dovresti rimuovere almeno \texttt{user} e \texttt{pass-obfuscated} (nome utente e password utilizzati per connettersi al server HylaFAX), \texttt{FromName}, \texttt{*ColState} (le larghezze delle colonne degli elenchi), \texttt{*Bounds} (la posizione delle varie finestre) e \texttt{mainwinLastTab}).

Nota che comunque l'utente può sempre modificare tali impostazioni per una data istanza di YajHFC. Esse sono reimpostate sui valori forzati solo quando YajHFC viene riavviato (in altre parole: l'utente può modificare tali impostazioni, ma esse non sono salvate tra diverse esecuzioni di YajHFC).


\subsection{Come posso evidenziare la cartella di HylaFAX \texttt{archive} in YajHFC?}

Dalla versione 0.4.0 è stato implementato il supporto per la cartella di HylaFAX \texttt{archive}.

Tuttavia non si può accedere a questa directory utilizzando la ``normale'' connessione ad HylaFAX come per le altre cartelle, poiché HylaFAX permette solamente di ottenere la lista delle sottocartelle nella directory \texttt{archive}, ma non di ottenere qualsiasi informazione (eccetto l'ID) riguardo i fax archiviati o i documenti allegati (se sai di una versione di HylaFAX dove ciò è differente, fammelo sapere).

Per questo motivo si deve accedere alla directory con altri metodi. Attualmente (0.4.0) YajHFC supporta l'accesso solo tramite file system. Questo significa che devi esportare la directory  \texttt{archive} sul server usando Samba, NFS o qualsiasi altro file system di rete, montarla sul client (se usi Unix; su Windows puoi utilizzare i percorsi UNC) e comunicare a YajHFC tra le Opzioni sotto quale percorso la directory \texttt{archive} si trova.

Quando lo hai fatto, l'elenco fax archiviati dovrebbe funzionare esattamente come gli altri.

\subsection{A cosa servono esattamente le diverse opzioni in \texttt{Percorsi \& Visualizzatori -> Impostazioni di visualizzazione ed invio} (da 0.4 in poi)?}

Per gli impazienti: le impostazioni consigliate sono (non sono predefinite perché richiedono \texttt{gs} e \texttt{tiff2pdf}):
\begin{itemize}
 \item \textbf{Formato:} PDF o TIFF
 \item \textbf{Invia file multipli come:} File singolo tranne la copertina
 \item \textbf{Visualizza fax come file singoli:} Sì
 \item \textbf{Visualizza/invia fax sempre in questo formato:} Sì
\end{itemize}

\subsubsection{Formato per visualizzazione/invio}

Il formato in cui i documenti dovrebbero essere convertiti se necessario. Generalmente PDF e TIFF danno risultati migliori che PostScript (dato che nell'ultimo caso si usa il device GhostScript \texttt{pswrite}).

\subsubsection{Invia file multipli come}

{\parindent0pt
\textbf{File multipli:}\\
Stesso comportamento delle versioni precedenti la 0.4.0. Se si allegano documenti multipli ad un singolo fax, tali documenti sono convertiti in PS o PDF, ma tenuti come file separati (es.: se invii un fax con i file \texttt{doc.ps} e \texttt{immagine.jpg}, viene fatto l'upload di due file separati)
\medskip

\textbf{File singolo tranne la copertina:}\\
Un unico file viene creato per l'intero fax, ma la copertina è tenuta in un file separato (es.: se invii un fax con \texttt{doc.ps} e \texttt{immagine.jpg}, viene creato e fatto l'upload di un singolo file PDF/PS/TIFF).\\
\textit{Vantaggio:} La risoluzione è ridotta a 196 dpi (-> upload di file più piccoli) e un singolo file può essere usato quando si invia un fax a destinazioni multiple.
\medskip

\textbf{Intero fax come file singolo:}\\
Un unico file viene creato per l'intero fax inclusa la copertina. Se il fax non ha copertina il comportamento è identico al caso precedente.\\
\textit{Vantaggio:} Nessuna conversione è necessaria sul client quando si visualizza il fax inviato.\\
\textit{Svantaggio:} Quando si invia un fax a più destinatari si deve creare e fare l'upload di un file per ogni destinatario.
}

\subsubsection{Visualizza fax come file singoli}
Se questa opzione è attiva e un fax sul server è composto da più file, viene creato un singolo file (sul client) per la visualizzazione.


\subsubsection{Visualizza/invia fax sempre in questo formato}
Questa opzione modifica il comportamento di ``Invia file multipli come'' e ``Visualizza fax come file singolo''.\\
Quando questa opzione è \textit{attivata}, un fax è convertito solo se è composto da più file. Se è composto da un solo file, il formato è lasciato invariato.\\
Quando questa opzione è \textit{disattivata}, anche un fax composto da un singolo file è convertito quando quel singolo file ha un formato differente da quello selezionato in ``Formato per visualizzazione/invio''.\\
\textit{Vantaggio:} Un singolo visualizzatore è utilizzato sia per i fax inviati che per quelli ricevuti (es.: per visualizzare in PDF i fax ricevuti).\\
\textit{Svantaggio:} Generalmente più conversioni di formato sono necessarie sul client.

\subsection{Quali caratteri sono riconosciuti come formato data/ora?}

La data è formattata utilizzando il formato Java \texttt{SimpleDateFormat}. Una descrizione dei caratteri riconosciuti si può trovare qui \url{http://java.sun.com/j2se/1.5.0/docs/api/java/text/SimpleDateFormat.html}.

\subsection{A cosa serve la funzione \texttt{Rispondi alla chiamata}?}

La stessa cosa del comando HylaFAX \verb#faxanswer#: comunica al server HylaFAX di provare a rispondere alla chiamata sul modem indicato anche se normalmente quel modem ignora le chiamate in ingresso. Questo può essere utile a scopo di test o per piccole installazioni dove un modem condivide la linea con un normale telefono.

\section{Problemi noti}

\subsection{Ho creato una copertina HTML, ma la formattazione in YajHFC appare errata!}

YajHFC utilizza il supporto HTML integrato in Java (\texttt{HTMLEditorKit} / \texttt{HTMLDocument}) per convertire HTML in PostScript. Tuttavia questo supporto è abbastanza limitato; supporta solo HTML 3.2.\\
Questo significa che layout complessi spesso non sono resi correttamente in YajHFC.
Per ottenere il layout desiderato, ci sono sostanzialmente le seguenti alternative:

\begin{itemize}
 \item Fai varie prove e tentativi fino a quando il layout si presenta bene (il pulsate anteprima nella finestra di invio fax mostra il layout dell'HTML convertito).
 \item Utilizza un editor HTML come Ekit (\url{http://www.hexidec.com/ekit.php}) che utilizzi anch'esso il supporto HTML di Java, in modo che lo stesso HTML sarà reso in modo simile in YajHFC.
 \item Utilizza un diverso formato per la copertina (come XSL:FO o ODT con il plugin FOP).
\end{itemize}

\subsection{Quando cerco di visualizzare i fax inviati ricevo sempre un errore che dice 
   \texttt{File format PCL not supported}, anche se tutti i documenti sono PostScript/PDF.}

Controlla che \texttt{Corregge problema in gestione file PCL} sia spuntato tra le Opzioni e riprova.

Alcune versioni di HylaFAX riportano erroneamente ``PCL'' come tipo file per tutti i documenti
associati ad un fax. Se tale impostazione è spuntata, YajHFC tenta di rilevare il tipo file
se esso viene riportato come PCL (normalmente funziona molto bene).

\subsection{Ricevo spesso un errore quando sto inviando due o più fax consecutivamente. Che posso fare?}

Alcune versioni di HylaFAX paiono avere problemi quando più di un fax è accodato in un'unica sessione.

Per evitare il problema, vai nella sezione \texttt{Server} nella finestra Opzioni, spunta \texttt{Crea una nuova sessione per ogni azione} e verifica se il problema persiste.
Se il problema \emph{non} si risolve, segnalami il bug con una e-mail.

\subsection{YajHFC sotto Windows alcune volte salva la sua configurazione in \texttt{C:\textbackslash .yajhfc} invece di \texttt{C:\textbackslash Documents and Settings\textbackslash USERNAME\textbackslash .yajhfc}}

Di default YajHFC salva le sue informazioni di configurazione nella sotto-directory \texttt{.yajhfc} della directory restituita dalla
proprietà di sistema JAVA \texttt{user.home}.
A volte alcune versioni di Java sembrano non impostare tale proprietà in modo corretto causando il problema appena descritto.

Come soluzione, puoi impostare tale proprietà in modo esplicito all'avvio di YajHFC utilizzando il parametro da riga di comando java \texttt{-D}, per esempio: \\
\texttt{java -Duser.home=\%USERPROFILE\% -jar "C:\textbackslash Programmi\textbackslash yajhfc.jar"}

\subsection{L'icona nell'area di notifica non viene evidenziata!}

A partire dalla versione 0.4.0 YajHFC supporta l'icona nell'area di notifica che viene evidenziata se si usa YajHFC con Java 1.6 (``Java 6'').
Se si usa Java 1.5 (``Java 5''), l'icona nell'area di notifica non è supportata.

Accertati quindi di avere installato Java 1.6. Se sei assolutamente certo di aver installato Java 1.6 e l'icona nell'area di notifica non viene evidenziata, segnalami il bug con una e-mail.


\section{Varie}

\subsection{Perché le password sono salvate in chiaro? (prima di 0.4)}

Detto semplicemente: perché non c'è alcun metodo che sia molto migliore.

YajHFC potrebbe codificare/``criptare'' le password in qualche modo prima di salvarle, ma se lo facesse, 
potresti sempre controllare il codice sorgente per scoprire come decriptarle
(anche se il codice sorgente di YajHFC fosse proprietario potresti sempre disassemblarlo
o fare vari esperimenti per scoprirlo).

L'unico metodo sicuro sarebbe di richiedere una master password ogni volta
che avvii YajHFC, ma secondo me questo non porterebbe ad un miglioramento rispetto
all'immettere la ``vera'' password.

A causa delle molte richieste le password sono offuscate usando un semplice algoritmo nelle versioni 0.4.0 e superiori.
Quanto detto sopra è comunque sempre vero, cioè una volta che si legge il codice sorgente la password può essere decriptata facilmente.

\subsection{Perché hai scelto un nome così stupido?}

YajHFC è iniziato come un progetto di test per Java e la libreria \texttt{gnu.hylafax} e quindi
non aveva un ``bel'' nome. Dopo averci lavorato per un po', ho notato che era diventato 
effettivamente utilizzabile, quindi ho deciso di dargli un nome.
Dato che in quel periodo stavo anche facendo esperimenti con il programma yast di SuSE e sapevo che
c'erano/ci sono già in giro molti altri client java per HylaFAX, l'ho semplicemente chiamato
``\textbf{y}et \textbf{a}nother \textbf{J}ava \textbf{H}ylaFAX \textbf{c}lient''. 

\end{document}
