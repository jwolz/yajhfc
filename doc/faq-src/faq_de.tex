\documentclass[a4paper,10pt,halfparskip,noparindent]{scrartcl}
\usepackage[latin1]{inputenc}
\usepackage{url}
\usepackage[left=3cm,right=3cm,top=2cm,bottom=2cm,nohead]{geometry}
\usepackage[ngerman]{babel}
\usepackage{ngerman}
\usepackage[colorlinks=false,pdfborder=0 0 0]{hyperref}
\usepackage[T1]{fontenc}
\usepackage{pslatex}

%opening
\title{YajHFC FAQ}
\author{}
\date{}
\begin{document}
\sloppy

\maketitle

\tableofcontents

\section{Installation}
\subsection{Woher bekomme ich einen Faxviewer für Windows?}

Im Allgemeinen können Sie irgendein beliebiges Programm benutzen, das TIFF-Dateien anzeigen kann.
Allerdings zeigen manche dieser Programme Faxe mit niedriger Auflösung gestaucht mit der falschen
(halben) Höhe an.

Glücklicherweise wird mit allen aktuellen Windowsversionen ein Programm mitgeliefert, das Faxe
korrekt anzeigen kann.

\begin{description}
\item[Windows 95/98/ME/2000:]
Sie können das unter Zubehör/Imaging zu findende Programm verwenden.
Um dieses in YajHFC zu verwenden, klicken sie auf die Durchsuchen-Schaltfläche des
\texttt{Befehlszeile für Faxviewer}-Eingabefelds, um \texttt{kodakimg.exe} auf ihrer Festplatte 
zu suchen und auszuwählen.\\
Diese Datei ist normalerweise entweder im Windows-Verzeichnis (Win 98/ME) \emph{oder} \\
in \verb.Programme\Windows NT\Zubehör\ImageVue. (Win 2000) zu finden.
       
\item[Windows XP:]
Sie können die integrierte "`Windows Bild- und Faxanzeige"' benutzen.
Um diese mit YajHFC zu benutzen, geben Sie bitte den folgenden Text in das \texttt{Befehlszeile für Faxviewer}-Eingabefeld ein:\\
\verb#rundll32.exe shimgvw.dll,ImageView_Fullscreen %s#
 \end{description}

      
\subsection{Woher bekomme ich einen Faxviewer für Linux/*BSD/... ?}
\texttt{kfax} funktioniert bei mir sehr gut, aber wie bei Windows können Sie auch
hier ein beliebiges Programm verwenden, das TIFF-Dateien anzeigen kann; durchsuchen
Sie am besten einfach die Paketdatenbank Ihrer Distribution.
(xloadimage funktioniert (zumindest bei mir) jedoch \emph{nicht}).


\subsection{Was sollte ich unter MacOS X als Faxviewer benutzen?}
Geben Sie einfach \verb.open %s. in das \texttt{Befehlszeile für Faxviewer}-Eingabefeld ein.
Die Faxe sollten nun mit der eingestellten Standardanwendung geöffnet werden.
Danke an Scott Harris für diese Info.

\subsection{Woher bekomme ich einen PostScript-Viewer?}
\begin{description}
\item[Windows:] Verwenden Sie am besten Ghostview von \url{http://www.cs.wisc.edu/~ghost/}
\item[Linux/*BSD/...:] Installieren Sie am besten irgendeines der PostScript-Viewer-Pakete
                (z.B. \texttt{gv, kghostview, gnome-gv, ...})
\end{description}

\subsection{Woher bekomme ich GhostScript?}
\begin{description}
\item[Windows:] Laden Sie es von \url{http://www.cs.wisc.edu/~ghost/} herunter
\item[Linux/*BSD/...:] 
Installieren Sie das GhostScript-Paket für Ihre Distribution (meistens wird es schon installiert sein; wenn nicht: der Paketname beginnt üblicherweise mit  \texttt{ghostscript} oder \texttt{gs})
\end{description}

\subsection{Woher bekomme ich TIFF2PDF?}
\begin{description}
\item[Windows:] 
Laden Sie es von \url{http://gnuwin32.sourceforge.net/downlinks/tiff.php} herunter.\\
Falls dieser Link nicht funktionert, sehen Sie bitte unter \url{http://gnuwin32.sourceforge.net/packages/tiff.htm} oder \url{http://www.libtiff.org/} nach.
\item[Linux/*BSD/...:] 
Installieren Sie das "`libtiff tools"'-Paket für Ihre Distribution. Normalerweise hat dieses Paket das Wort \texttt{tiff} in seinem Namen (unter Debian/Ubuntu heißt es \texttt{libtiff-tools} und unter SUSE \texttt{tiff}).
\end{description}

\subsection{Wie kann ich dem Windows-Setup bereits heruntergeladene Versionen von tiff2pdf und GhostScript übergeben?}

Beginnend mit Version 0.4.2 kann das Windows-Setup optional tiff2pdf und GhostScript herunterladen und installieren.
Bevor ein Download gestartet wird, sucht das Setupprogramm an folgenden Stellen nach bereits heruntergeladenenen Versionen:
\begin{enumerate}
 \item Dem Verzeichnis, in dem sich \texttt{setup.exe} befindet
 \item Dem Desktop des angemeldeten Benutzers (wo auch die Dateien gespeichert werden, wenn sich entscheiden, die heruntergeladenen Dateien zu sichern)
\end{enumerate}

Bitte beachten Sie, dass die Dateinamen exakt denen der heruntergeladenen Dateien entsprechen müssen (d.h.  \texttt{gs864w32.exe} (32 bit) oder \texttt{gs864w64.exe} (64 bit) bzw. \texttt{tiff-3.8.2-1.exe}).

\section{Benutzung des Programms}	

\subsection{Wie kann ich PostScript-Deckblattvorlagen bearbeiten?}
Die Vorlagen müssen in dem gleichen speziellen PostScript-Format vorliegen,
wie es auch von dem HylaFAX-\texttt{faxcover}-Programm verwendet wird.
Schauen Sie bitte auf den folgenden Seiten nach, um Hinweise zu erhalten, wie 
man solche Dateien anlegen bzw. bearbeiten kann (auf Englisch):\\
\url{http://www.hylafax.org/HylaFAQ/Q202.html}\\
\url{http://www.hylafax.org/howto/tweaking.html}\\

Alternativ können Sie ab YajHFC 0.3.7 auch Deckblätter im HTML-Format benutzen, oder mit Hilfe eines Plugins im XSL:FO- und ODT-Format.

\subsection{Welche Felder werden in einem HTML-Deckblatt erkannt?}

Die folgenden "`Wörter"' werden (ohne Berücksichtigung der Groß-/""Kleinschreibung) bei Verwendung einer HTML-Datei als Deckblatt durch die entsprechenden Werte ersetzt:

\begin{center}
\begin{tabular}{|l|l|}
\hline
\bfseries Wort & \bfseries Bedeutung \\
\hline\hline
\ttfamily @@Name@@ & Der Name des Empfängers \\\hline
\ttfamily @@Location@@ &  Der Ort des Empfängers \\\hline
\ttfamily @@Company@@ &  Die Firma des Empfängers \\\hline
\ttfamily @@Faxnumber@@ &  Die Faxnummer des Empfängers \\\hline
\ttfamily @@Voicenumber@@ &  Die Telefonnummer des Empfängers \\\hline
\ttfamily @@FromName@@ & Der Name des Senders \\\hline
\ttfamily @@FromLocation@@ &  Der Ort des Senders \\\hline
\ttfamily @@FromCompany@@ &  Die Firma des Senders \\\hline
\ttfamily @@FromFaxnumber@@ &  Die Faxnummer des Senders \\\hline
\ttfamily @@FromVoicenumber@@ &  Die Telefonnummer des Senders \\\hline
\ttfamily @@FromEMail@@ &  Die E-Mail-Adresse des Senders \\\hline
\ttfamily @@Subject@@ & Der Betreff des Faxes \\\hline
\ttfamily @@Date@@ & Das heutige Datum \\\hline
\ttfamily @@PageCount@@ & Die Seitenanzahl \textit{ohne} das Deckblatt selbst \\\hline
\ttfamily @@Comments@@ & Die für dieses Fax eingegebenen Kommentare \\\hline
\end{tabular}
\end{center}

Beginnend mit Version 0.4.0, sind die folgenden zusätzlichen Felder verfügbar (beachten Sie, dass die kursiven Felder leer sind, wenn Sie "`Fax erneut senden"' verwenden):
\begin{center}
\begin{tabular}{|l|p{.7\textwidth}|}
\hline
\bfseries Wort & \bfseries Bedeutung \\
\hline\hline
\ttfamily @@Surname@@ & Der Nachname des Empfängers (erneut gesendete Faxe: gleich wie \texttt{@@Name@@}) \\\hline
\ttfamily\itshape @@GivenName@@ & Der Vorname des Empfängers \\\hline
\ttfamily\itshape @@Title@@ & Der Titel des Empfängers \\\hline
\ttfamily\itshape @@Position@@ & Die Position des Empfängers \\\hline
\ttfamily\itshape @@Department@@ & Die Abteilung des Empfängers\\\hline
\ttfamily @@CompanyName@@ & Der Firmenname des Empfängers (ohne Abteilung) (erneut gesendete Faxe: gleich wie  \texttt{@@Company@@})\\\hline
\ttfamily\itshape @@Street@@ & Die Straße des Empfängers \\\hline
\ttfamily @@Place@@ & Der Ort des Empfängers (ohne PLZ oder Straße) (erneut gesendete Faxe: gleich wie \texttt{@@Location@@})\\\hline
\ttfamily\itshape @@ZIPCode@@ & Die Postleitzahl des Empfängers \\\hline
\ttfamily\itshape @@State@@ & Die Region bzw. der Bundesstaat des Empfängers\\\hline
\ttfamily\itshape @@Country@@ & Das Land des Empfängers\\\hline
\ttfamily\itshape @@EMail@@ & Die E-Mail-Adresse des Empfängers\\\hline
\ttfamily\itshape @@WebSite@@ & Die Website des Empfängers\\\hline\hline
\ttfamily @@FromSurname@@ & Der Nachname des Senders \\\hline
\ttfamily @@FromGivenName@@ & Der Vorname des Senders \\\hline
\ttfamily @@FromTitle@@ & Der Titel des Senders \\\hline
\ttfamily @@FromPosition@@ & Die Position des Senders \\\hline
\ttfamily @@FromDepartment@@ & Die Abteilung des Senders\\\hline
\ttfamily @@FromCompanyName@@ & Der Firmenname (ohne Abteilung) des Senders\\\hline
\ttfamily @@FromStreet@@ & Die Straße des Senders \\\hline
\ttfamily @@FromPlace@@ & Der Ort des Senders (ohne PLZ oder Straße)\\\hline
\ttfamily @@FromZIPCode@@ & Die Postleitzahl des Senders \\\hline
\ttfamily @@FromState@@ & Die Region bzw. der Bundesstaat des Senders\\\hline
\ttfamily @@FromCountry@@ & Das Land des Senders\\\hline
\ttfamily @@FromEMail@@ & Die E-Mail-Adresse des Senders\\\hline
\ttfamily @@FromWebSite@@ & Die Website des Senders\\\hline
\ttfamily @@TotalPageCount@@ & Die Seitenanzahl \textit{inklusive} des Deckblatts \\\hline
\ttfamily @@CCNameAndFax@@ & Die Namen und Faxnummern der anderen Empfänger dieses Faxes im Format \textit{Name1 <Faxnummer1>; Name2 <Faxnummer2>; ...} (nur Versionen > 0.4.4)\\\hline
\end{tabular}
\end{center}

Diese Ersetzung wird auf der Quelltextebene durchgeführt, so dass diese Wörter vermutlich nicht mehr erkannt werden, wenn sich die Formatierung innerhalb eines Wortes ändert (z.B. \texttt{@@sub\textit{ject@@}}).

Beginnend mit Version 0.4.2 werden einige einfache bedingte Anweisungen unterstützt. Diese werden über HTML-Kommentare implementiert, also sollten Sie sicherstellen, dass der Text innerhalb einer bedingten Anweisung keine solchen Kommentare enthält.
\begin{center}
\begin{tabular}{|l|p{.5\textwidth}|}
\hline
\bfseries Wort & \bfseries Bedeutung \\
\hline\hline
\ttfamily @@IfSomeFilled:\textit{Feld1,Feld2,...}@@ & Den folgenden Text nur einschließen, wenn mindestens eines der angegebenen Felder gefüllt ist (d.h. einen Wert mit einer Länge > 0 besitzt).\\\hline
\ttfamily @@IfAllFilled:\textit{Feld1,Feld2,...}@@ & Den folgenden Text nur einschließen, wenn alle der angegebenen Felder gefüllt sind (d.h. einen Wert mit einer Länge > 0 besitzen).\\\hline
\ttfamily @@IfSomeEmpty:\textit{Feld1,Feld2,...}@@ & Den folgenden Text nur einschließen, wenn mindestens eines der angegebenen Felder leer ist (d.h. einen Wert mit einer Länge = 0 besitzt).\\\hline
\ttfamily @@IfAllEmpty:\textit{Feld1,Feld2,...}@@ & Den folgenden Text nur einschließen, wenn alle der angegebenen Felder leer sind (d.h. einen Wert mit einer Länge = 0 besitzen).\\\hline
\ttfamily @@Else@@ & Den folgenden Text nur einschließen, wenn die Bedingung der letzten \texttt{@@If...@@}-Anweisung nicht wahr war.\\\hline
\ttfamily @@EndIf@@ & Markiert das Ende des von einem \texttt{If} beeinflussten Bereichs. Jedes \texttt{@@If...@@} muss genau ein entsprechendes \texttt{@@EndIf@@} besitzen.\\\hline
\end{tabular}
\end{center}

\subsection{Woher kann ich das alte Standarddeckblatt bekommen, das vor Version 0.4.2 verwendet wurde?}

Sie können es im "`Coverpage examples"'-Archiv herunterladen von: \url{http://download.yajhfc.de/misc/coverpages.zip}


\subsection{Mir gefiel der alte Sendedialog besser. Kann ich ihn zurück haben?}

Öffnen Sie einfach das Optionen-Dialogfeld und wählen sie \texttt{Traditionell} als Stil des Sendedialogs aus.

\subsection{Ich möchte ein Telefonbuch über JDBC ansprechen, aber YajHFC kann den Treiber nicht finden, obwohl 
ich einen korrekten "`class path"' angegeben habe.}

Wenn das \texttt{-jar} Befehlszeilenargument angegeben wird, ignoriert Java einen angegebenen benutzerdefinierten
class path.
Starten Sie YajHFC in diesem Fall daher mit den folgenden Befehlen (\texttt{/pfad/zum/db-treiber.jar} und \texttt{/pfad/zu/yajhfc.jar} sind selbstverständlich durch die entsprechenden Dateinamen (mit Pfad) zu ersetzen ):
\begin{description}
\item [Linux/Unix:] \verb#java -classpath /pfad/zum/db-treiber.jar:/pfad/zu/yajhfc.jar yajhfc.Launcher#
\item [Windows:] \verb#java -classpath c:\pfad\zum\db-treiber.jar;c:\pfad\zu\yajhfc.jar yajhfc.Launcher#
\end{description}

\subsection{Was kann beim \texttt{passt auf}-Operator im "`Benutzerdef. Filter"'-Dialog als Wert eingegeben werden?}

Reguläre Ausdrücke (regular expressions). Eine Kurzreferenz (auf Englisch) über die verwandte Syntax 
kann auf folgender Seite gefunden werden: 
\url{http://java.sun.com/j2se/1.5.0/docs/api/java/util/regex/Pattern.html}

Bitte beachten Sie, dass Reguläre Ausdrücke etwas anderes sind als Wildcards: 
Beispielsweise müssen Sie, um den Effekt des \verb.*.-Wildcards zu erreichen, \verb#.*# eingeben und 
für den Effekt des \verb#?#-Wildcards \verb#.#.

\subsection{Welche Kommandozeilenargumente werden von YajHFC unterstützt?}

Ausgabe von \verb#java -jar yajhfc.jar --help# (Version 0.4.4):
\begin{verbatim}
Aufruf:
java -jar yajhfc.jar [OPTIONEN]... [ZU SENDENDE DATEIEN]...

Argumentbeschreibung:
-r, --recipient=EMPFÄNGER             Gibt einen Faxempfänger an. Sie können
                                      entweder eine Faxnummer oder detaillierte
                                      Deckblattinformationen angeben (siehe die
                                      FAQ für das Format im letzteren Fall).
                                      Sie können --recipient mehrmals angeben
                                      für mehrere Empfänger.
-C, --use-cover[=yes|no]              Benutze ein Deckblatt beim Versendes des
                                      Faxes.
-s, --subject=BETREFF                 Der Betreff des Faxes für das Deckblatt.
    --comment=KOMMENTAR               Der Kommentar für das Deckblatt.
-M, --modem=MODEM                     Setzt das zum Senden des Faxes zu
                                      verwendende Modem. Geben Sie entweder den
                                      Namen des Modems (z.B. ttyS0) an oder
                                      "any", um ein beliebiges Modem zu
                                      verwenden.
    --stdin                           Lese die zu sendende Datei von der
                                      Standardeingabe.
-A, --admin                           Starte im Admin-Modus.
-d, --debug                           Gib einige Debugging-Informationen aus.
-l, --logfile=LOGDATEI                Die Logdatei, in die die
                                      Debugginginformationen ausgeben (falls
                                      nicht angegeben, benutze die
                                      Standardausgabe).
    --appendlogfile=LOGDATEI          Hänge die Debugginginformationen an die
                                      angegebene Logdatei an.
    --background                      Falls noch keine Instanz von YajHFC
                                      läuft, erstelle eine neue Instanz und
                                      beende diese Instanz (nachdem das zu
                                      sendende Fax abgeschickt wurde)
    --noclose                         YajHFC nach Absenden des Faxes nicht
                                      schließen.
    --no-wait                         Warte nicht darauf, dass der Sendedialog
                                      geschlossen wird. Wenn YajHFC mehrfach
                                      aufgerufen wird, bevor der Benutzer den
                                      Sendedialog schließt, werden die
                                      angegebenen Dokumente zur Liste der
                                      Dateien hinzugefügt. Diese Option
                                      impliziert --background.
-T, --showtab=0|R|1|S|2|T             Setzt den beim Start anzuzeigenden Tab.
                                      Geben Sie 0 oder R für den Tab
                                      "Empfangen", 1 oder S für "Gesendet" oder
                                      2 oder T für "Sendend" an.
    --windowstate=N|M|I|T             Setzt den anfänglichen Status des
                                      Hauptfensters auf _N_ormal, _M_aximiert,
                                      als _I_con (minimiert) oder minimiert in
                                      die System-_T_ray.
    --loadplugin=JARDATEI             Gibt die JAR-Datei eines zu ladenden
                                      YajHFC-Plugins an.
    --loaddriver=JARDATEI             Gibt die JAR-Datei eines zu ladenden
                                      JDBC-Treibers an.
    --override-setting=SCHLÜSSEL=WERT Überschreibt den Wert der angegebenen
                                      Einstellung für diese Sitzung. Die
                                      überschriebene Einstellung wird nicht
                                      gespeichert.
    --no-plugins                      Deaktiviert das Laden von Plugins aus der
                                      plugin.lst-Datei.
    --no-gui                          Sende ein Fax mit einer nur minimalen
                                      graphischen Oberfläche.
    --no-check                        Unterdrückt die Prüfung der Javaversion
                                      beim Start.
-c, --configdir=VERZEICHNIS           Setzt ein anstatt von ~/.yajhfc zu
                                      verwendendes Konfigurationsverzeichnis.
-h, --help[=SPALTEN]                  Zeigt diesen Text (ggf. formatiert für
                                      SPALTEN Spalten) an.
\end{verbatim}


\subsection{Wie kann ich mittels des \texttt{-{-}recipient}-Parameters Informationen für das Deckblatt übergeben?}

Beginnend mit Version 0.4.0 können Sie solche Informationen über \texttt{Name:Wert}-Paate, getrennt durch Semikolons, übergeben.
Zum Beispiel, um ein Fax zu "`Max Mustermann"' in "`Bad Musterstadt"' mit der Faxnummer 0123456 zu senden, benutzen Sie bitte die folgende Kommandozeile:

\texttt{java -jar yajhfc.jar \textit{[...]} -{-}recipient=\dq givenname:Max;surname:Mustermann;location:Bad Musterstadt;faxnumber:0123456\dq \textit{[...]}}

Die folgenden Feldnamen werden erkannt:
\begin{center}
\begin{tabular}{|l|p{.7\textwidth}|}
\hline
\bfseries Field name & \bfseries Meaning \\
\hline\hline
\ttfamily surname & Der Nachname des Empfängers \\\hline
\ttfamily givenname & Der Vorname des Empfängers \\\hline
\ttfamily title & Der Titel des Empfängers \\\hline
\ttfamily position & Die Position des Empfängers \\\hline
\ttfamily department & Die Abteilung des Empfängers\\\hline
\ttfamily company & Der Firmenname des Empfängers\\\hline
\ttfamily street & Die Straße des Empfängers \\\hline
\ttfamily location & Der Ort des Empfängers\\\hline
\ttfamily zipcode & Die PLZ des Empfängers \\\hline
\ttfamily state & Der Bundesstaat bzw. die Region des Empfängers\\\hline
\ttfamily country & Das Land des Empfängers\\\hline
\ttfamily email & Die E-Mail-Adresse des Empfängers\\\hline
\ttfamily faxnumber & Die Faxnummer des Empfängers \\\hline
\ttfamily voicenumber & Die Telefonnummer des Empfängers \\\hline
\ttfamily website & Die Website des Empfängers\\\hline
\end{tabular}
\end{center}

\subsection{Was bedeutet die Tabellenspalte XYZ?}

Höchstwahrscheinlich weiß ich das auch nicht so genau, da ich die
Spaltenbeschreibungen einfach aus der \verb.faxstat(1).-man page (JobFmt/RcvFmt)
herauskopiert und nach bestem Wissen ggf. abgekürzt und übersetzt habe.

\subsection{Wie kann ich Standardeinstellungen festlegen?}

Beginnend mit Version 0.4.0 werden die folgenden Dateien (falls sie existieren) geladen, um die gespeicherten Einstellungen wiederherzustellen:
\begin{enumerate}
 \item \texttt{[Verzeichnis, in dem sich yajhfc.jar befindet]/settings.default}
 \item die Benutzereinstellungen aus \texttt{\{user.home\}\footnote{Unter Windows ist \texttt{user.home} üblicherweise \texttt{C:\textbackslash Dokumente und Einstellungen\textbackslash BENUTZERNAME}.}/.yajhfc/settings} (falls \texttt{-{-}configdir=DIR} angegeben wurde, wird stattdessen \texttt{DIR/settings} verwendet)
 \item \texttt{[Verzeichnis, in dem sich yajhfc.jar befindet]/settings.override}
\end{enumerate}

Die Einstellungen aus später geladenenen Dateien überschreiben dabei die Einstellungen aus den früher geladenen Dateien, d.h. die Einstellungen aus \texttt{settings.override} haben Priorität vor den Benutzereinstellungen und vor \texttt{settings.default}.
\medskip

Diese Logik kann dazu verwendet werden, Standardeinstellungen festzulegen (z.B. innerhalb eines Netzwerks):\\
Konfigurieren Sie einfach eine YajHFC"=Installation nach Wunsch, kopieren Sie dann \texttt{\{user.home\}/.yajhfc/settings} in das Verzeichnis, in dem \texttt{yajhfc.jar} ist und bennenen Sie diese Datei in \texttt{settings.default} um.
\medskip

Auf ähnliche Weise können "`Overrides"' (Einstellungen, die die Benutzereinstellungen immer überschreiben) festgelegt werden. In diesem Fall ist es allerdings zu empfehlen, die \texttt{settings}"=Datei zu editieren (sie ist eine einfache Textdatei) und alle Zeilen für Einstellungen, die sie nicht überschreiben möchten, zu entfernen (üblicherweise betrifft dies zumindest \texttt{user} und \texttt{pass-obfuscated} (Benutzername und Passwort für die Verbindung zum HylaFAX"=Server), \texttt{FromName}, \texttt{*ColState} (die Breiten der Tabellenspalten), \texttt{*Bounds} (die Größe und Position der verschiedenen Fenster) und \texttt{mainwinLastTab}).

Beachten Sie allerdings, dass der Benutzer diese Einstellungen in einer laufenden YajHFC"=Instanz immer noch auf einen anderen Wert setzen kann. Sie werden erst wieder auf die "`Override"=Werte"' gesetzt, wenn YajHFC neu gestartet wird (in anderen Worten: der Benutzer kann diese Einstellungen setzen, aber Sie werden zwischen zwei Starts von YajHFC nicht gespeichert).


\subsection{Wie kann ich das HylaFAX \texttt{archive}"=Verzeichnis in YajHFC anzeigen?}

Mit Version 0.4.0 wurde eine Unterstützung für das HylaFAX \texttt{archive}"=Verzeichnis hinzugefügt.

Auf dieses Verzeichnis kann allerdings nicht wie bei den anderen Verzeichnisse über die "`normale"' HylaFAX"=Verbindung zugegriffen werden, da mit HylaFAX nur eine Liste der Unterverzeichnisse unterhalb des \texttt{archive}"=Verzeichnisses geholt werden kann, aber keine weiteren Information (außer der ID) über die archivierten Aufträge oder die zugehörigen Dokumente abgerufen werden können (falls Sie eine HylaFAX"=Version kennen, die dies ermöglicht, lassen Sie es mich bitte wissen).

Aus diesem Grund muss auf dieses Verzeichnis auf einem anderen Weg zugegriffen werden.
Im Moment (0.4.0) unterstützt YajHFC nur den Zugriff über das Dateisystem. Das heißt, dass Sie das \texttt{archive}"=Verzeichnis auf dem Server mittels Samba, NFS oder irgendeinem anderen Netzwerkdateisystem freigeben müssen, es auf dem Klienten einbinden (falls Sie Unix verwenden; unter Windows können Sie auch einfach UNC"=Pfade verwenden) und in den Optionen von YajHFC einstellen, unter welchem Pfad das \texttt{archive}"=Verzeichnis gefunden werden kann.

Sobald dies erledigt wurde, sollte die Archiv"=Tabelle genau wie die anderen Tabellen funktionieren.


\subsection{Was bewirken die verschiedenen Optionen unter \texttt{Pfade \& Betrachter -> Einstellungen für Betrachten und Senden} (0.4 aufwärts)?}

Für Ungeduldige: Die empfohlenen Einstellungen sind (da diese \texttt{gs} und \texttt{tiff2pdf} benötigen, sind sie allerdings nicht voreingestellt):
\begin{itemize}
 \item \textbf{Format:} PDF oder TIFF
 \item \textbf{Sende mehrere Dateien als:} Einzelne Datei ohne Deckblatt
 \item \textbf{Zeige als einzelne Datei an:} Ja
 \item \textbf{Zeige/sende immer in diesem Format:} Ja
\end{itemize}

\subsubsection{Format zum Betrachten/Senden}

Das Format, in das die Dokumente falls nötig konvertiert werden sollen. Im Allgemeinen funktionieren PDF und TIFF hier deutlich besser als PostScript (da für das letztere GhostScripts \texttt{pswrite}"=Device verwendet wird).

\subsubsection{Sende mehrere Dateien als}

{\parindent0pt
\textbf{Mehrere Dateien:}\\
Das selbe Verhalten wie Versionen vor 0.4.0. Wenn Sie mehrere Dateien an einen einzelnen Faxauftrag anhängen, werden diese Dokumente in PS oder PDF konvertiert, aber weiterhin als separate Dateien gehalten (z.B. wenn Sie ein Fax mit den Dateien \texttt{doc.ps} und \textbf{picture.jpg} senden, werden zwei Dateien hochgeladen).
\medskip

\textbf{Einzelne Datei ohne Deckblatt:}\\
Für das gesamte Fax wird eine einzelne Datei erstellt, das Deckblatt aber als separate Datei gehalten (z.B. wenn Sie ein Fax mit den Dateien \texttt{doc.ps} und \textbf{picture.jpg} senden, wird eine einzelne PDF/PS/TIFF"=Datei erstellt und hochgeladen).
\textit{Vorteil:} Die Auflösung wird auf 196dpi reduziert (-> kleinere Dateien/kleinerer Upload) und eine einzige Dokumentendatei kann bei einem Fax an mehrere Empfänger verwendet werden.
\medskip

\textbf{Komplettes Fax als einzelne Datei:}\\
Für das gesamte Fax wird eine einzige Datei einschließlich des Deckblatts erstellt. Falls das Fax kein Deckblatt hat, verhält sich diese Einstellung gleich wie oben.\\
\textit{Vorteil:} Beim Betrachten des Faxes sind keine Konvertierungen auf den Klienten nötig.\\
\textit{Nachteil:} Wenn ein Fax an mehrere Empfänger gesendet wird, muss für jeden Empfänger eine Datei erzeugt und hochgeladen werden.
}

\subsubsection{Zeige Fax als einzelne Datei an}

Falls diese Einstellung aktiviert ist und das Fax aus mehreren Dateien besteht, wird zum Betrachten eine einzige Datei erzeugt (auf dem Klienten).


\subsubsection{Betrachte/Sende Faxe immer in diesem Format}

Diese Option modifiziert das Verhalten von "`Sende mehrere Dateien als"' und "`Zeige Fax als einzelne Datei an"'.\\
Wenn diese Option \textit{deaktiviert} ist, wird ein Fax nur konvertiert, wenn es aus mehreren Dateien besteht. Wenn es aus einer einzelnen Datei besteht, wird das Format beibehalten.\\
Wenn diese Option \textit{aktiviert} ist, wird ein aus einer einzelnen Datei bestehendes Fax auch dann konvertiert, wenn sich das Format dieser Datei von dem bei "`Format zum Betrachten/Senden"' ausgewählten unterscheidet.\\
\textit{Vorteil:} Sowohl für gesendete als auch für empfangene Faxe wird der selbe Betrachter benutzt (z.B. um auch empfangene Faxe als PDF anzuzeigen).\\
\textit{Nachteil:} Üblicherweise sind auf dem Klienten mehr Konvertierungen nötig.


\subsection{Welche Zeichen werden beim Datums- und Uhrzeitformat erkannt?}

Die Datumsformatierung erfolgt durch ein Java"=\texttt{SimpleDateFormat}. Eine Beschreibung der erkannten Buchstaben kann unter \url{http://java.sun.com/j2se/1.5.0/docs/api/java/text/SimpleDateFormat.html} (Englisch) gefunden werden.

\subsection{Was tut die \texttt{Anruf beantworten} Funktion?}

Das selbe wie das HylaFAX \verb#faxanswer# Kommando: Es instruiert den HylaFAX"=Server, zu versuchen, auf dem angegebenen Modem einen Anruf zu beantworten, auch wenn das Modem normalerweise eingehende Anrufe ignoriert.
Dies kann für Testzwecke nützlich sein oder bei kleinen Installationen, wo sich ein Modem eine Leitung mit einem normalen Telefon teilt.

\section{Probleme/Bekannte Fehler}

\subsection{Ich habe ein HTML-Dokument/Deckblatt erstellt, aber die Formatierung in YajHFC ist nicht korrekt!}

YajHFC verwendet die in Java integrierte HTML-Unterstützung (\texttt{HTMLEditorKit} / \texttt{HTMLDocument}), um HTML in PostScript zu konvertieren. Allerdings ist diese Unterstützung relativ eingeschränkt, insbesondere unterstützt sie nur HTML 3.2.\\
Dies bedeutet, dass komplexe Layouts in YajHFC vermutlich häufig nicht korrekt dargestellt werden. Um das gewünschte Layout zu erreichen, gibt es im Grunde die folgenden Möglichkeiten:

\begin{itemize}
 \item Probieren Sie solange herum ("`trial \& error"'), bis das Layout richtig aussieht (der Vorschau-Knopf im Sendedialog zeigt das Layout der konvertierten HTML-Datei an).
 \item Verwenden Sie einen HTML-Editor wie Ekit (\url{http://www.hexidec.com/ekit.php}), der ebenfalls die Java-HTML-Unterstützung verwendet, so dass die Darstellung ähnlich zu der in YajHFC ist.
 \item Verwenden Sie ein anderes Format für das Deckblatt (z.B. XSL:FO oder ODT mit dem FOP Plugin).
\end{itemize}


\subsection{Wenn ich Versandte Faxe anzeigen möchte, erhalte ich immer die Fehlermeldung 
\texttt{Dateiformat PCL wird nicht unterstützt}, obwohl alle Dokumente im PostScript/PDF-Format sind.}

Aktivieren Sie bitte das Kontrollfeld \texttt{PCL-Dateityp-Bugfix verwenden} im Optionen-Dialogfeld und
versuchen Sie es nocheinmal.

Einige HylaFAX-Versionen liefern inkorrekterweise den Dateityp "`PCL"' bei allen
mit einem Faxauftrag verbundenen Dokumenten zurück.
Wenn Sie dieses Kontrollfeld aktivieren, versucht YajHFC den Dateityp selbst herauszufinden,
wenn als Typ PCL zurückgegeben wird (was üblicherweise recht gut funktioniert).


\subsection{Ich bekomme häufig eine Fehlermeldung, wenn ich zwei oder mehr Faxe hintereinander sende.}

Manche Versionen des HylaFAX"=Servers haben scheinbar Probleme, wenn mehr als ein Fax innerhalb einer Sitzung gesendet wird.

Um dieses Problem zu umgehen, öffnen Sie bitte den Tab "`Server"' im Optionendialogfeld, aktivieren das Kontrollkästchen "`Erstelle neue Sitzung für jede Aktion"' und probieren aus, ob das Problem weiterhin besteht.
Falls dies auch nichts hilft, schreiben Sie mir bitte einen Bug Report.


\subsection{Unter Windows speichert YajHFC seine Konfiguration manchmal unter \texttt{C:\textbackslash .yajhfc} und nicht unter \texttt{C:\textbackslash Dokumente und Einstellungen\textbackslash BENUTZERNAME\textbackslash .yajhfc}}

In der Standardeinstellung speichert YajHFC seine Konfigurationsinformationen im Unterverzeichnis \texttt{.yajhfc} des Verzeichnisses, das von der Java-System-Eigenschaft \texttt{user.home} zurückgeliefert wird.
Einige Javaversionen scheinen diese Eigenschaft manchmal nicht korrekt zu setzen, was zu dem oben beschriebenen Fehlverhalten führt.

Um diesen Fehler zu umgehen, können Sie diese Eigenschaft mittels des \texttt{-D}-Kommandozeilenarguments von Java explizit setzen, z.B.: \\
\texttt{java -Duser.home=\%USERPROFILE\% -jar \dq C:\textbackslash Programme\textbackslash yajhfc.jar\dq}

\subsection{Das Tray"=Icon wird nicht angezeigt!}

Ab version 0.4.0 unterstützt YajHFC ein Tray"=Icon, das angezeigt wird, wenn YajHFC unter Java 1.6 ("`Java 6"') ausgeführt wird. Wenn Sie Java 1.5 ("`Java 5"') verwenden, wird ein Tray"=Icon leider nicht unterstützt.

Stellen Sie also bitte sicher, dass Java 1.6 installiert ist. Wenn Sie sich absolut sicher sind, dass Java 1.6 installiert ist und das Tray"=Icon immer noch nicht angezeigt wird, senden Sie mir bitte einen Bug"=Report.


\section{Verschiedenes}

\subsection{Warum werden Passwörter im Klartext gespeichert? (vor 0.4)}

Kurz gesagt: Weil es keine Möglichkeit gibt, die wirklich besser wäre.

YajHFC könnte die Passwörter natürlich irgendwie verschleiern/kodieren/"`verschlüsseln"',
bevor sie gespeichert werden, aber wenn es das täte, könnte man immer in den Quelltext schauen,
um herauszufinden, wie man diese wieder entschlüsselt (selbst wenn YajHFC Closed Source-Software wäre,
könnte man es immer noch disassemblieren oder einfach etwas herumexperimentieren, um das
herauszufinden).

Die einzig wirklich sichere Methode würde es erfordern, beim Start von YajHFC immer ein "`Master-Passwort"'
einzugeben, was meiner Meinung auch nicht einfacher/besser wäre, als jedesmal gleich das "`echte"' Passwort
einzugeben.

Aufgrund vieler Nachfragen werden Passwörter ab Version 0.4.0 mittels eines einfachen Algorithmus verschleiert.
Dies bedeutet allerdings nicht, dass das oben geschriebene nicht mehr stimmt, d.h. nach Anschauen des Quelltexts lassen sich die Passwörter auf einfache Weise entschlüsseln.


\subsection{Warum wurde dieser seltsame Name gewählt?}

YajHFC war ursprünglich nur ein Testprojekt, um Java und die \texttt{gnu.hylafax}-Bibliothek besser
kennenzulernen und hatte noch keinen "`schönen"' Namen.
Als ich bisschen daran gearbeitet hatte, stellte ich fest, dass als Ergebnis in der Tat ein benutzbares
Programm herausgekommen war, also beschloss ich dem Ganzen einen Namen zu geben.
Da ich zur selben Zeit auch etwas mit SuSEs yast herumspielte und ich natürlich wusste, dass es schon sehr viele
andere Java-HylaFAX-Clients gab und gibt, nannte ich das Programm einfach "`noch ein Java-HylaFAX-Client"'
(\textbf{y}et \textbf{a}nother \textbf{J}ava \textbf{H}ylaFAX \textbf{c}lient).



\end{document}
