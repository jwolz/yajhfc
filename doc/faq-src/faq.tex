\documentclass[a4paper,10pt]{scrartcl}
\usepackage[latin1]{inputenc}
\usepackage{url}
\usepackage[left=3cm,right=3cm,top=2cm,bottom=2cm,nohead]{geometry}
\usepackage[colorlinks=false,pdfborder=0 0 0]{hyperref}

%opening
\title{YajHFC FAQ}
\author{}
\date{}
\begin{document}

\maketitle

\tableofcontents

\section{Installation}
\subsection{Where do I get a fax viewer for Windows?}

Generally you can use any program that can display TIFF files, although 
some of them display faxes in low resolution with an incorrect (half) height.

Luckily, all recent Windows versions come with program that does this correctly:

\begin{description}
\item[Windows 95/98/ME/2000:]
 You can use the Accessories/Imaging application.
      To use it with YajHFC, use the browse button of the \texttt{Command line for fax viewer}
      box to browse for \texttt{kodakimg.exe} on your hard drive.\\
      This file usually resides either in the Windows directory (Win 98/ME) \emph{or} \\
      in \verb.Program Files\Windows NT\Accessories\ImageVue. (Win 2k).
       
\item[Windows XP:]
You can use the integrated ``picture and fax viewer'' (sorry, I don't know
      how this is exactly called in English).
      To use it, enter the following text into the \texttt{Command line for fax viewer} box:\\
      \verb#rundll32.exe shimgvw.dll,ImageView_Fullscreen %s#
 \end{description}

      
\subsection{Where do I get a fax viewer for Linux/*BSD/... ?}
\texttt{kfax} works very well for me, but just as with Windows, you may use any program that 
can display TIFF files, just search your distribution's package database 
(xloadimage does \emph{not} work for me, however).


\subsection{What should I use as a fax viewer on Mac OS X?}
Just enter \verb.open %s. into the \texttt{Command line for fax viewer} box.
The faxes should open in the default application now.
Thanks to Scott Harris for figuring this out.


\subsection{Where do I get a PostScript viewer?}
\begin{description}
\item[Windows:] Use Ghostview from \url{http://www.cs.wisc.edu/~ghost/}
\item[Linux/*BSD/...:] Just install one of the Postscript viewer packages
                (e.g. \texttt{gv, kghostview, gnome-gv, ...})
\end{description}
		
\section{Usage of the program}	

\subsection{How can I edit cover page templates?}
The templates have to be in the same special PostScript format the HylaFAX
\texttt{faxcover} program uses. See the following pages for hints how to create/edit
such files: \\
\url{http://www.hylafax.org/HylaFAQ/Q202.html}\\
\url{http://www.hylafax.org/howto/tweaking.html}\\

\subsection{I want to access a phone book over JDBC, but YajHFC won't find the driver even if I specify a correct class path when invoking java.}

If you use the \texttt{-jar} command line argument, java ignores a user defined class path.
So, please start YajHFC using the following commands (replace \texttt{/path/to/dbdriver.jar} and \texttt{/path/to/yajhfc.jar} with the respective real paths and file names of course):
\begin{description}
\item [Linux/Unix:] \verb#java -classpath /path/to/dbdriver.jar:/path/to/yajhfc.jar yajhfc.Launcher#
\item [Windows:] \verb#java -classpath c:\path\to\dbdriver.jar;c:\path\to\yajhfc.jar yajhfc.Launcher#
\end{description}

\subsection{What can I enter as a value for the \texttt{matches} Operator in the custom filter dialog?}

Regular Expressions. A short reference about the accepted syntax can be found at:
\url{http://java.sun.com/j2se/1.5.0/docs/api/java/util/regex/Pattern.html}

Please note that Regular Expressions are not the same as wildcards: 
For example, to get the effect of the \verb.*. wildcard, you have to use \verb#.*# and 
for the effect of the wildcard \verb#?# use \verb#.#.

\subsection{Which command line arguments does YajHFC understand?}

\begin{verbatim}
General usage:
java -jar yajhfc.jar [--help] [--debug] [--admin] [--background|--noclose]
         [--configdir=directory] [--loadplugin=filename]
         [--showtab=0|R|1|S|2|T] [--recipient=...] [--stdin | filename ...]
Argument description:
filename     One or more file names of PostScript files to send.
--stdin      Read the file to send from standard input
--recipient  Specifies the phone number of a recipient to send the fax to.
             You may specify multiple arguments for multiple recipients.
--admin      Start up in admin mode
--debug      Output some debugging information
--background If there is no already running instance, launch a new instance
             and terminate (after submitting the file to send)
--noclose    Do not close YajHFC after submitting the fax
--showtab    Sets the tab to display on startup. Specify 0 or R for the "Received",
             1 or S for the "Sent" or 2 or T for the "Transmitting" tab.
--loadplugin Specifies the jar file of a YajHFC plugin to load
--configdir  Sets a configuration directory to use instead of ~/.yajhfc
--help       Displays this text
\end{verbatim}


\subsection{When trying to view sent faxes I always get an error message saying 
   \texttt{File format PCL not supported}, although all the documents are PostScript/PDF.}

Check the \texttt{Use PCL file type bugfix} checkbox in the Options dialog and try again.

Some HylaFAX versions incorrectly report a file type of ``PCL'' for all documents
associated with a job. If this checkbox is checked, YajHFC tries to guess the
file type if PCL is reported (which usually works pretty well).

\subsection{YajHFC running under Windows sometimes saves its configuration in \texttt{C:\textbackslash .yajhfc} instead of \texttt{C:\textbackslash Documents and Settings\textbackslash USERNAME\textbackslash .yajhfc}}

As a default YajHFC saves its configuration information in the subdirectory \texttt{.yajhfc} of the directory returned by
the Java system property \texttt{user.home}.
Sometimes some Java versions seem to not set this property correctly leading to the misbehaviour described above.

As a workaround, you can set this property explicitely when starting YajHFC by using java's \texttt{-D} command line switch, for example: \\
\texttt{java -Duser.home=\%USERPROFILE\% -jar "C:\textbackslash Program Files\textbackslash yajhfc.jar"}

\subsection{What does the column XYZ mean?}

Most likely, I don't know exactly either, because the column descriptions
are simply copied from the \verb.faxstat(1). man page (JobFmt/RcvFmt) and 
abbreviated/translated.

\section{Miscellaneous}

\subsection{Why are passwords stored in clear text?}

Simply put: Because there isn't any method that is much better.

YajHFC could encode/``encrypt'' them somehow before storing, but if it did, 
you could always look at the source code to find out how to decrypt them
(even if YajHFC was closed source software you could still disassemble it
or experiment a bit to find that out).

The only secure method would require you to enter a master password every time
you start YajHFC, but in my opinion that would not provide an improvement over
entering the ``real'' password.


\subsection{Why did you choose that stupid name?}

YajHFC started out as a test project for Java and the \texttt{gnu.hylafax} library and so
it didn't have a ``nice'' name. After working at it for a while, I noticed that it turned 
out to be actually useable, so I chose to give it a name.
Because I was also playing around with SuSE's yast at that point of time and I knew there
were/are already a lot of other java hylafax clients out there, I simply called it
``\textbf{y}et \textbf{a}nother \textbf{J}ava \textbf{H}ylaFAX \textbf{c}lient''. 

\end{document}
